\chapter{Appendices}

\section{Appendix A}
\label{appendix:a}

%--------------------------------------------------------------------------------------
\begin{figure}[b]
    \centering
    \def\svgwidth{.95\linewidth}
    \import{resources/Figures}{moire_displacement.pdf_tex}
    \caption{Result of performing CBED shift analysis on the region of a moiré lattice highlighted in blue in Figure \ref{fig:dub_moire}. \textbf{a,b,c,d}) Highlight the shift in the y-direction, the shift in the x-direction, the magnitude of the shift, and, the angle of the displacement respectively. All values are given are (sub)pixel shifts across the sensor. The axes correspond with those given in the vHAADF image \textbf{e}.}
    \label{fig:m_dis}
\end{figure}
%--------------------------------------------------------------------------------------

\begin{figure}[H]
    \centering
    \def\svgwidth{.5\linewidth}
    \import{resources/Figures}{moire_efield_charge.pdf_tex}
    \caption{\textbf{top}) Reconstructed high-angle annular dark field photograph of two moiré unit cells. \textbf{middle}) Momentum transferred to the passing electron probe by the sample. \textbf{bottom}) Electric field direction and charge density that would create the electric field such that the momentum transferred to the electron probe would be the same as in the \textbf{middle} plot.}
    \label{fig:m_mom}
\end{figure}

\section{Appendix B}
\label{appendix:b}
\begin{figure}[H]
    \centering
    \def\svgwidth{.7\linewidth}
    \import{resources/Figures}{trip_moire_cbed_shift.pdf_tex}
    \caption{Analysis of the convergent beam electron diffraction pattern shift of the sample presented in Section \ref{sec:large_moire}}
    \label{fig:appendix_cbed_trip_moire}
\end{figure}


\subsection*{}
