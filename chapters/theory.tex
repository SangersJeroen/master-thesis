\section{Theory}
\subsection{TMDC/Crystal structure}
\subsubsection{Transition metal dichalcogenides}
\subsubsection{Crystal lattice and band structure}
An infinitely repeating group of atoms is called an ideal crystal, such a crystal is constructed by attaching the same group of atoms, often called a unit cell, to a lattice.
The lattice can be constructed from $n$ independent lattice vectors. $n=1$ for an atomic chain, $n=2$ for a two-dimensional monolayer, and, $n=3$ for a three-dimensional crystal.
If no smaller repeating group of atoms can be constructed to fill the lattice then this group of atoms is called the primitive unit cell and the $n$-independent lattice vectors are then called the primitive translation vectors $a_{n}$ \cite{Kittel1995-qt}.
Each of the $n$ lattice vectors signifies a direction and length of displacement needed such that the shifted crystal lattice is indistinguishable from the original crystal lattice \ref{eq:lattice_equivalent}.
Lattice vectors are also used to specify the orientation of a crystal plane by denoting where the plane intersects the lattice vector, this procedure allows for unique indexing of crystallographic planes. The use of these planes will be discussed in \ref{sec:diffraction}.
\begin{equation}
    \vec{r}' = \vec{r} + u_1 \vec{a_1} +u_2 \vec{a_2} + u_3 \vec{a_3}
    \label{eq:lattice_equivalent}
\end{equation}

\subsubsection{Reciprocal lattice and electron diffraction}
\label{sec:diffraction}
In the previous section the crystal lattice was introduced, and it was mentioned that there were unique planes characterized by the points where they intersect the lattice vectors.
In reciprocal space every lattice point is equivalent to one set of these planes.
To best understand a crystal, it is helpful to conceptualize it as having two lattices. The first lattice pertains to the organization of the atoms within the crystal's unit cells. The second lattice is a pattern of points that is specific to each crystal and does not correspond to the atom arrangement. Rather, each point in the lattice is linked to a particular set of planes within the crystal \cite{Williams2009-ww}.
Both lattice constructions are equally valid but are helpful under different circumstances; the reciprocal lattice, for instance, is a useful geometrical construct when talking about diffraction.

The reciprocal lattice, just like the crystal lattice, is constructed by vectors; in the case for the reciprocal lattice these are the reciprocal lattice vectors $\vec{b}_n$.
The reciprocal lattice vectors are constructed from the real-space lattice vectors using equation (\ref{eq:lattice_ortho_norm}) and satisfy relation (\ref{eq:lattice_vec_prop}) with their real-space counterpart.
Using these definitions the reciprocal lattice vectors are unique.
Any reciprocal vector can now be composed uniquely by a linear combination of the reciprocal lattice vectors, such that any vector is scaled and summed. If the scalars are integers they are the miller indices and correspond to a crystallographic plane. \\

\begin{minipage}{0.5\textwidth}
    \begin{equation}
        \vec{b}_i = 2 \pi \vec{a}_j \times \vec{a}_k \cdot \left[ \vec{a}_i \cdot ( \vec{a}_j \times \vec{a}_k ) \right]^{-1} 
    \label{eq:lattice_ortho_norm}
   \end{equation}
\end{minipage}%
\begin{minipage}{0.5\textwidth}
    \begin{equation}
        \vec{b}_i \cdot \vec{a}_j = 2\pi \delta_{ij}
    \label{eq:lattice_vec_prop}
   \end{equation}
\end{minipage}


\begin{enumerate}
    \item crystal lattice
    \item diffraction pattern / reciprocal space
\end{enumerate}
\subsection{Moiré}
\begin{enumerate}
    \item moire pattern -> stacking
    \item lattice relaxation
    \item mini brillouin zone / aligned or antialigned
    \item hybridisation, inter/intra transistions and excitons
    \item band bending types, umklapp,  flat bands
\end{enumerate}


