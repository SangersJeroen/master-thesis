\chapter{Conclusions and Outlook}
\label{sec:outlook}
This study has successfully demonstrated the production of large-scale \SI{1}{\micro\meter} heterobilayer structure using the stamping and mechanical transfer methods described herein. While achieving precise control over the moiré angle remains an ongoing challenge, the process shows immense promise for rapid prototyping of heterostructures. This potential extends to variations in material composition, layer thickness, moiré twist angles, and possibly even controlled strain.\\

Our spectroscopic techniques have proven effective in identifying materials and determining their thickness, facilitating the characterisation and selection of transition metal dichalcogenide flakes. Additionally, the refinement of wet transfer techniques has led to improved success rates in transferring heterostructures from silicon to holey carbon grids, indicating a positive trajectory for future applications.\\

A key finding of this research is the use of the Electron Microscope Pixel Array Detector (EMPAD) in characterising heterostrain in stacked heterostructures, even those with with larger field-of-view and a moiré angle of \SI{12.2}{\degree}. The strain analysis technique, employing the exit-wave power cepstrum transform, has shown to be highly effective, particularly in addressing complex features in stamped and stacked heterostructures. \\

Additionally, the centre-of-mass analysis applied to EMPAD data has opened up a new possibility for characterising the electric and magnetic potentials of our stamped heterostructures at the moiré length scales, effectively disentangling signals from features of varying sizes. Further experimentation and thoughtful consideration will be necessary to develop the expertise and intuition required to select the feature size to capture during data acquisition effectively.\\

In conclusion, the methodologies presented in this report establish a foundation for the prototyping of novel heterostructures and the exploration of relationships between physical characteristics, such as strain and twist angles, and the electronic and magnetic properties of these materials. Looking ahead, the integration of machine learning into the analysis of the vast quantities of data collected by EMPAD holds great promise. It will enable the design and fabrication of more sophisticated quantum systems.
