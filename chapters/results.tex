\section{Case study of $MoSe_2$/$WSe_2$ moiré heterostructures}

\subsection{Verification of moiré pattern and its parameters}

\begin{figure}
    \centering
    \def\svgwidth{.9\linewidth}
    \import{resources/Figures}{moire_flakes.pdf_tex}
    \caption{(\textbf{a}, \textbf{d}, \textbf{h}) HR-TEM images of three different $MoSe_2$/$WSe_2$-Moiré heterostructures stamped under different angles verified by the FFT of images (\textbf{b}, \textbf{e}, \textbf{i}) taken at $460k\times$ magnification (not shown). The Moiré angle can be determined by measurement of the angle in between the two diffraction peaks in a set, for \textbf{b}, \textbf{e}, \textbf{i} the angle was measured to be $16.6\degree$, $12.2\degree$, and, $8.3\degree$ respectively. (\textbf{c}, \textbf{f}, \textbf{j}) displaying a reconstruction of filtered diffraction patterns, showing a clear Moiré superlattice. The Moiré superlattice cell was measured to be \SI{0.47}{nm}, \SI{1.06}{nm}, and, \SI{1.69}{nm}; for \textbf{c}, \textbf{f}, and, \textbf{j}. (\textbf{a}) The two layers of material show a rough surface that seems to arise during or after transfer to the TEM grid.}
    \label{fig:moire_overview}
\end{figure}

The samples displayed in Figure \ref{fig:moire_overview} have been fabricated using the method outlined in Section \ref{sec:fab_method} but had not been checked under the spectroscope as it had not arrived by then. The flakes were selected for stamping only by using a optical microscope and identifying thin regions. As can be seen in Figure \ref{fig:moire_overview}\textbf{a,d,h} not all flakes are equally thin and are thus not all composed of two layers of monolayer material. The high-resolution TEM images do however show that using the stamping method it is possible to create heterostructures comprised of monolayer material. Figures \ref{fig:moire_overview}\textbf{b,c,f} show the fast Fourier transforms of HRTEM images taken at $460\mathrm{k}\times$ and clearly depict the characteristic double set of Bragg spots that indicate the two crystal layers. Masking the diffraction peaks and taking the inverse Fourier transform then creates a noise-free representation of the moiré lattice (Figure \ref{fig:moire_overview}\textbf{c,f,j}), which shows the expected relation of increasing unit cell for decreasing moiré angles.


\subsection{Analysis of in-plane strain}
\begin{figure}
    \centering
    \def\svgwidth{.74\linewidth}
    \import{resources/Figures}{strain_overview.pdf_tex}
    \caption{Overview of the heterostrain present in both the top and bottom layer of the heterostructure displayed in Figure \ref{fig:moire_overview}\textbf{d} for the left and right column respectively. The red box denotes the reference region for the strain analysis. In any white regions not enough peaks were found to perform strain analysis. \textbf{a,b,e,f}) Show compressive or tensile strain in the top layer for \textbf{a, b} and in the bottom layer for \textbf{e, f}. \textbf{c, g}) illustrate the shear strain in both layers. \textbf{e, h} show the local rotation of the layer with respect to the average rotation in the red box.}
    \label{fig:strain_overview}
\end{figure}

For the analysis of heterostrain the heterostructure displayed in Figure \ref{fig:moire_overview}\textbf{d} was selected as its transfer to the TEM grid went without issue thus leaving no ambiguity in material, whereas for the other two heterostructures depicted in Figure \ref{fig:moire_overview} there were a lot of flakes twirling around in the IPA when dissolving the PVA substrate. The two layers were stamped one on top of another at an angle of \SI{12.2}{\degree} creating a moiré lattice with \SI{1.06}{\nano\meter} periodicity.
The heterostrain analysis was carried out using the exit-wave power cepstrum transform method. Both layers were analysed separately by tracking the corresponding EWPC peaks such that only the strain and rotation within each single layer are considered. In Figure \ref{fig:strain_overview} the results are presented for each layer in their own column. The rows, from top to bottom, denote the compressive/tensile strain in the $x$ and $y$ direction, the shear strain, and, the rotation of the layer; all relative to the reference area denoted by the red square. The rotation in the bottom row is not the same as the moiré angle under which the layers are stamped. The white regions denote the places where the to be tracked peaks were not found, and thus no information could be retrieved.
In Figure \ref{fig:strain_overview}\textbf{a-c, e-d} it is shown that both layers are effected by between \SI{-5}{\percent} and \SI{5}{\percent} strain mostly concentrated near the edges of the supporting carbon mesh over which both layers are draped and for the top layer there also exist strain close to a vertically extending fold in the layer. The relative rotation plot for the top layer shows a stark difference in relative rotation between the left and right side of the fold, meaning that the rotation likely pushed the material in on itself creating the fold. The relative rotation tapers off as the fold also disappears when moving away from the flakes edge as can be seen in Figure \ref{fig:moire_overview}\textbf{b}.\\
\newpage

\section{Case study: An analysis of features in stamped $WSe_2$}
\begin{figure}
    \centering
    \def\svgwidth{.95\linewidth}
    \import{resources/Figures}{e2_overview.pdf_tex}
    \caption{\textbf{a}) High-angle annular dark field photograph, image shows material increasing in layer number from the bottom left to the top right. These layers were slightly rotated during transfer, leading to the formation of the moiré pattern. Across this material another thin strip has settled during transfer causing a further modulation of the moiré period. \textbf{b}) High-resolution TEM image of the region highlighted in \textbf{a} with a moiré unit cell outlined. \textbf{c}) FFT of the high-resolution image showing a \SI{2.3}{\degree} rotation between the strip and larger layers.}
    \label{fig:dub_moire}
\end{figure}

This sample of stamped $WSe_2$ was created using the described mechanical transfer technique and was originally considered unsuccessful as the stamped flakes got jostled up during transfer negating any effort put into alignment of the flakes. Upon inspection in STEM mode the large features highlighted with the green outline in Figure \ref{fig:dub_moire}\textbf{a} appeared and are the effect of three different layers of material slightly out of alignment with one another. The resulting moiré unit cell is highlighted with the dashed outline in Figure \ref{fig:dub_moire}\textbf{b} and is measured to be caused a \SI{2.3}{\degree} rotation between the thin strip's crystal lattice (highlighted in green) and the moiré pattern present in the underlying layers (highlighted in blue).
In the following sections three regions from the previously introduced sample will be looked at: a hole in otherwise uniform material to verify the centre-of-mass analysis tools, a large periodicity moiré lattice to see what can be expected, and finally, the double misaligned region.

\subsection{Hole}

\begin{figure}
    \centering
    \def\svgwidth{.95\linewidth}
    \import{resources/Figures}{hole_displacement.pdf_tex}
    \caption{Result of performing CBED shift analysis on a large hole in otherwise uniform material. \textbf{a,b,c,d}) Highlight the shift in the y-direction, the shift in the x-direction, the magnitude of the shift, and, the angle of the displacement respectively. All values are given are (sub)pixel shifts across the sensor. The axes correspond with those given in the vHAADF image \textbf{e}.}
    \label{fig:hole_dis}
\end{figure}

As highlighted in the previous theory section on centre-of-mass (COM) analysis there are two different scenarios in which one needs to consider different effects on the diffracted electron beam and one thus needs to consider different analysis approaches to the COM analysis. One such scenario is one in which the electron probe's cross-sectional area is much smaller than the feature being imaged, which is the case for the hole in the otherwise uniform moiré bilayer depicted in Figure \ref{fig:hole_dis}\textbf{e}. As reported in other works, changes in sample geometry such  as edges and slopes can deflect the electron bundle. This deflection is towards the thicker region in the sample \cite{ophusFourDimensionalScanningTransmission2019a,dekkers1974differential}. By tracking the edge of the bright field disk and its deviation with respect to its mean position it is possible to recover the deflection of the diffraction pattern in both the $x$- and $y$-direction. This deflection in the $x$- and $y$-deflection is plotted in Figure \ref{fig:hole_dis}\textbf{a,b} and the combined deflection magnitude and deflection angle in Figure \ref{fig:hole_dis}\textbf{c,d} respectively. The effects are as expected with the electron beam displacements being towards the material. There also appear to be a few probe points in the centre of the hole were barely any displacement is measured indicating that the beam's cross-sectional area is indeed smaller than that of the hole. Using the information on the displacement of the diffraction pattern it is also possible to compute the momentum transfer imparted on the electron beam using similar methods as described in \cite{mullerAtomicElectricFields2014}. The net momentum transfer and its direction is plotted in Figure \ref{fig:hole_mom} as well as a virtual-HAADF image for reference. The plot shows higher momenta transfer for probe positions nearer to the hole's edge and tapers of as the probe moves further over the sample or more towards the centre of the hole.

\begin{figure}
    \centering
    \def\svgwidth{.7\linewidth}
    \import{resources/Figures}{hole_momentum_transfered.pdf_tex}
    \caption{Left panel show a virtual-HAADF image taken by masking the dataset. Right panel shows the momentum transfer imparted on the electron beam by the sample, the electron beam gets deflected towards the thicker material.}
    \label{fig:hole_mom}
\end{figure}

\subsection{Small Moiré}

\begin{figure}
    \centering
    \def\svgwidth{.95\linewidth}
    \import{resources/Figures}{moire_displacement.pdf_tex}
    \caption{Result of performing CBED shift analysis on the region of a moiré lattice highlighted in blue in Figure \ref{fig:dub_moire}. \textbf{a,b,c,d}) Highlight the shift in the y-direction, the shift in the x-direction, the magnitude of the shift, and, the angle of the displacement respectively. All values are given are (sub)pixel shifts across the sensor. The axes correspond with those given in the vHAADF image \textbf{e}.}
    \label{fig:m_dis}
\end{figure}

Data for the following moiré lattice was collected in the blue highlighted region in Figure \ref{fig:moire_overview} and is estimated to have a \SI{\leq 1}{\degree} moiré angle as there were no discernable peaks in the FFT of the HRTEM image in that region. Similarly to the dataset of the hole the diffraction pattern shift was calculated for the probe positions and was then corrected for in the dataset to realign all the bright field disks. The diffraction pattern shift is plotted in Figure \ref{fig:m_dis}. A clear distinction between unit cell interior and unit cell walls can be seen as both features appear to displace the electron beam diffraction pattern in opposite directions. After aligning all the bright field disks it is now possible to calculate the COM from the intensity distribution solely within the bright field disk, doing so leads to the results presented in Figure \ref{fig:m_mom}; in which are presented, form top to bottom: a virtual-HAADF image made by masking the dataset, the magnitude and direction of momentum transfer in the sample plane, and, the distribution of charges that would lead to such a momentum transfer. Even though two moiré unit cells are plotted there seem to be no discernable repeating features.

\begin{figure}
    \centering
    \def\svgwidth{.5\linewidth}
    \import{resources/Figures}{moire_efield_charge.pdf_tex}
    \caption{}
    \label{fig:m_mom}
\end{figure}

\subsection{Large Moiré}
% Vortices at the vertices

\begin{figure}
    \centering
    \def\svgwidth{.95\linewidth}
    \import{resources/Figures}{trip_moire_displacement.pdf_tex}
    \caption{Results from the centre-of-mass analysis performed bright field disk of the double moiré region highlighted in Figure \ref{fig:dub_moire} after the electron diffraction pattern shift was accounted for. \textbf{a,b,c,d}) Highlight the shift in the y-direction, the shift in the x-direction, the magnitude of the shift, and, the angle of the displacement respectively. All values are given are (sub)pixel shifts across the sensor. The axes correspond with those given in the vHAADF image \textbf{e}.}
    \label{fig:trip_m_dis}
\end{figure}

The double moiré pattern observed in the green highlighted region of Figure \ref{fig:moire_overview}\textbf{a} consists of three different layers with two distinct moiré angles between them, one very close to zero degrees and the other approximately \SI{2.3}{\degree} creating a moiré unit cell roughly \SI{10}{\nano\meter} in size. From the virtual-HAADF image of this dataset the probe cross-sectional area was assumed to be roughly the same size as roughly half a moiré unit cell. Therefore, the CBED shift was compensated for and COM analysis was performed on the bright field disk as an attempt to peek inside the moiré unit cell. The results of the CBED shift analysis are displayed in Appendix \ref{appendix:a}.
The resulting COM analysis for the intensity redistribution due to electric and/or magnetic fields is/are plotted in Figure \ref{fig:trip_m_dis}. A clear periodicity matching the visual details from the virtual-HAADF image can be observed in all the displacement measurements, with good agreement in values between moiré unit cells. Figure \ref{fig:trip_m_dis}\textbf{c} shows the magnitude of the COM displacement, which shows a clear distinction between the high symmetry points that show up as white blobs in the vHAADF image (Fig. \ref{fig:trip_m_dis}\textbf{e}) and the thin semi-disordered regions connecting them. The magnitude of the COM redistribution increases radially outwards from the centre of a high symmetry region. To investigate further the calculated momentum transfer is pictured in Figure \ref{fig:dub_moire} and highlights the same effect more clearly. For reference the outline of the moiré unit cell is drawn and encloses two high symmetry points with the third placed on the vertex. The two vortices enclosed are higher in magnitude than the vortex at the vertex even though they are not equal in intensity in the vHAADF image. This phenomenon is at the time of writing not yet fully understood and requires further though, theoretical calculations and analysis.


\begin{figure}
    \centering
    \def\svgwidth{.9\linewidth}
    \import{resources/Figures}{trip_moire_momentum.pdf_tex}
    \caption{Image shows the momentum transfer imparted on the electron beam by features in the sample. The moiré unit cell is highlighted and the same as in Figure \ref{fig:dub_moire}\textbf{b}.}
    \label{fig:trip_m_mom}
\end{figure}

