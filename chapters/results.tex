\section{Case study of $MoSe_2$/$WSe_2$ moiré heterostructures}

\subsection{Verification of moiré pattern and its parameters}

\begin{figure}[t]
    \centering
    \def\svgwidth{.9\linewidth}
    \import{resources/Figures}{moire_flakes.pdf_tex}
    \caption{(\textbf{a}, \textbf{d}, \textbf{h}) HR-TEM images of three different $MoSe_2$/$WSe_2$-Moiré heterostructures stamped under different angles verified by the FFT of images (\textbf{b}, \textbf{e}, \textbf{i}) taken at $460k\times$ magnification (not shown). The Moiré angle can be determined by measurement of the angle in between the two diffraction peaks in a set, for \textbf{b}, \textbf{e}, \textbf{i} the angle was measured to be $16.6\degree$, $12.2\degree$, and, $8.3\degree$ respectively. (\textbf{c}, \textbf{f}, \textbf{j}) displaying a reconstruction of filtered diffraction patterns, showing a clear Moiré superlattice. The Moiré superlattice cell was measured to be \SI{0.47}{nm}, \SI{1.06}{nm}, and, \SI{1.69}{nm}; for \textbf{c}, \textbf{f}, and, \textbf{j}. (\textbf{a}) The two layers of material show a rough surface that seems to arise during or after transfer to the TEM grid.}
    \label{fig:moire_overview}
\end{figure}
% Figure above a) Sample C1, d) C2, h) C5.




\subsection{Analysis of in-plane strain}
\begin{figure}[t]
    \centering
    \def\svgwidth{.74\linewidth}
    \import{resources/Figures}{strain_overview.pdf_tex}
    \caption{Overview of the heterostrain present in both the top and bottom layer of the heterostructure displayed in Figure \ref{fig:moire_overview}\textbf{d} for the left and right column respectively. The red box denotes the reference region for the strain analysis. In any white regions not enough peaks were found to perform strain analysis. \textbf{a,b,e,f}) Show compressive or tensile strain in the top layer for \textbf{a, b} and in the bottom layer for \textbf{e, f}. \textbf{c, g}) illustrate the shear strain in both layers. \textbf{e, h} show the local rotation of the layer with respect to the average rotation in the red box.}
    \label{fig:strain_overview}
\end{figure}

Will this text force the figure to show? let's hope so!
If not then the figures can be found after the references for some reason
\newpage
