\section{Advanced electron microscopy techniques to map moiré physics at the nanoscale; an electron microscope pixelated array detector.}

\subsection{Electron microscope pixel-array detector}
%In TEM operating mode a parallel electron beam is used to illuminate the sample and form an image on either the phosphorous screen or digital camera, this spreads the beam over a larger area such that the local electron dose is relatively uniform.

\begin{figure}[h]
	\centering
	\def\svgwidth{0.5\linewidth}
	\import{resources/Figures}{empad-haadf.pdf_tex}
	\caption{Figure schematically shows the difference between a monolithic fixed geometry high-angle annular dark field (HAADF) detector and a pixelated-array detector on the left- and right-hand side of the image respectively. As shown, the HAADF detector is only able to resolve features whose scattering diffracts the beam onto the monolithic annular detector. Since the Bragg angle is fixed for each feature, the signal can only be captures at a certain camera length. The HAADF detector is a single large punctured "pixel" that bunches all incident signals together. In contrast, the EMPAD is able to capture signals diffracted over a wider range of angles for every camera length as well as spatially separate the different signals, a HAADF and bright field detector could then be simulated by virtually masking and integrating the equivalent regions of the data after a measurement.}
	\label{fig:empad_haadf_comparison}
\end{figure}
The electron microscope pixel-array detector, colloquially called the EMPAD for short, is as the name suggest a sensor for electron microscopes that consists of a grid of direct electron detecting pixels. Even though the EMPAD can be used in standard transmission electron imaging its strengths lie in scanning-TEM imaging due to the relatively little pixels but greater dynamic range of said pixels.
In a scanning-TEM (STEM) mode the beam is cone-shaped and converges to a thin point-like spot that scans over the specimen, for small semi-convergence angles the zeroth order diffraction disk is a magnitude higher in intensity than the first order Bragg reflection disks. Since there is valuable information in the distribution of intensity in all disks its important that the detector has sufficient dynamic range to count individual electrons at all intensity ranges simultaneously using a high-gain per pixel charge integration circuit. The second major advantage of the EMPAD is the pixelated-array of detectors that, contrary to the regular monolithic annular dark field detectors, preserves the deflection angle information of the transmitted electron beams.
Traditionally, elastic scattering deflects the electrons from the optical axis onto one of the annular ring shaped detectors encircling the optical axis, these detectors are called the dark field detectors. These singular annular detectors have a fixed detector geometry that only captures signals at certain camera lengths when diffracted beams are impinging on the detector, as schematically shown in Figure \ref{fig:empad_haadf_comparison}. The EMPAD solves this problem by measuring the whole convergent-beam electron-diffraction (CBED) pattern using a single fast-readout, high dynamic range pixel grid on which every pixels' electron dose is stored separately such that after acquisition of a complete STEM scan the bright- and dark-field detectors can be virtually recreated by integrating the electron dose using annular or circular masks on the data.
The pixelated nature of the detector enables the precise computation of the intensity distribution of not only the whole CBED patten but also within each diffraction disk within the CBED pattern, greatly improving the potential resolution by means of ptychography methods \cite{pennycookEfficientPhaseContrast2015, yangEfficientPhaseContrast2015a} as well as enabling charge density analysis \cite{hachtelSubAngstromElectricField2018,wenMapping1DConfined2022,fangAtomicElectrostaticMaps2019} and strain analysis \cite{hanStrainMappingTwoDimensional2018, ophusFourDimensionalScanningTransmission2019}.
The EMPAD's sensor is made up of a grid of 128 by 128 pixels and will thus always image the CBED or NBED pattern at this resolution. Every pixel counts single electron charges and stores the count in a 32-bit number. The EMPAD is able to equal the field-of-view of the HAADF-detector with a maximum of 512 by 512 equally positioned scan points. A schematic showing the geometry of a measurement is given in Figure \ref{fig:4d_dataset}a where a scanning probe is shown to scan over a diamond shaped moiré unit cell creating a CBED pattern on the EMPAD's pixel-array sensor. Nine CBED patterns for differing probe positions are displayed in Figure \ref{fig:4d_dataset}b. By virtually masking and integrating the first-order Bragg reflections for every CBED pattern to a single electron count one is able to recover a virtual annular dark-field image.

\begin{figure}[h]
	\centering
	\def\svgwidth{1\linewidth}
	\import{resources/Figures}{4d-dataset.pdf_tex}
    \caption{\textbf{a}) Schematic overview of an EMPAD measurement where a convergent electron beam probes the sample at positions $\vec{r}_1$ through $\vec{r}_5$ producing convergent-beam electron diffraction (CBED) pattern images of 128 by 128 pixels for each such position.\textbf{b}) Nine such CBED patterns imaged at different positions within a moiré cell using electric field measurement conditions.\textbf{c}) A virtual annular dark field image made by masking a thin annulus in the 4d-dataset. With only the first order Bragg reflections captured by the mask the atomic periodicity is visible within the image.}
	\label{fig:4d_dataset}
\end{figure}

\subsection{Strain analysis}
Strain analysis in electron microscopy can be achieved through a multitude of ways, the main distinction between all of them is whether they are performed by analysing a real-space image captured by high-resolution TEM techniques or performed by analysing reciprocal space images captured through scanning-TEM techniques. An example of the first method is geometric phase analysis where displacements of atoms in a crystal are directly observed in the high-resolution image and strain can be computed \cite{HYTCH1998131, hytchGEOMETRICPHASEANALYSIS, nguyenAtomicDefectsDoping2017}. The second method relies primarily on the fact that a CBED/NBED pattern for a small enough convergent electron probe directly measures the local crystal structure. The reciprocal-space unit cell of the local crystal structure and thus the positions of the peak in the CBED pattern are directly correlated with the size of the real-space unit cell such that a compressing force in the real-space unit cell will elongate the reciprocal-space unit cell in the same direction and vice-versa for a tensile strain \cite{ophusFourDimensionalScanningTransmission2019, vanwinkleRotationalDilationalReconstruction2023, kazmierczakStrainFieldsTwisted2021, hanStrainMappingTwoDimensional2018}, tracking the peaks then gives access to the strain information.  In the following sections the second method will be applied for both large and small field-of-views.

\subsubsection{Micrometer field-of-view strain analysis}
\begin{figure}[h]
	\centering
	\def\svgwidth{0.9\linewidth}
	\import{resources/Figures}{adf_cbed_ewpc.pdf_tex}
    \caption{\textbf{a}) virtual annular dark-field image reconstructed by masking the dataset such that the detector geometry is equivalent to that of the HAADF-detector. Image is taken at the transition region of monolayer $MoSe_2$ (bottom) to a heterostructure of $MoSe_2$ and $WSe_2$ (top) on a lacy carbon grid. The faint circles in the middle of the image are contamination from inspection in TEM mode. Contamination in STEM mode caused the bright white spots on the left side of the image. \textbf{b}) logarithm of the nanobeam electron-diffraction pattern captured in the heterobilayer region of the image in \textbf{a}. Two sets of six diffraction disks, one set of six for each material, can be seen in the pattern. Due to local tilt of the crystalline layers diffraction disks on one side of the central $000$-disk, in this case the bottom, are brighter than disks on the opposing side. \textbf{c}) Exit-wave power cepstrum (EWPC) transform of the NBED pattern displayed in \textbf{b}. The EWPC pattern does not suffer from unequal intensity distributions. Peaks in the pattern are distinguishable in the EWPC transform.}
	\label{fig:adf_nbed_ewpc}
\end{figure}



\subsubsection{Nanometre field-of-view strain analysis}

\subsection{Ptychography}

\subsection{Charge density analysis}

