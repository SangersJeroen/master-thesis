\section{Case study of $MoSe_2$/$WSe_2$ moiré heterostructures}

\subsection{Verification of moiré pattern and its parameters}

\begin{figure}[h]
    \centering
    \def\svgwidth{.9\linewidth}
    \import{resources/Figures}{moire_flakes.pdf_tex}
    \caption{(\textbf{a}, \textbf{d}, \textbf{h}) HR-TEM images of three different $MoSe_2$/$WSe_2$-Moiré heterostructures stamped under different angles verified by the FFT of images (\textbf{b}, \textbf{e}, \textbf{i}) taken at $460k\times$ magnification (not shown). The Moiré angle can be determined by measurement of the angle in between the two diffraction peaks in a set, for \textbf{b}, \textbf{e}, \textbf{i} the angle was measured to be $16.6\degree$, $12.2\degree$, and, $8.3\degree$ respectively. (\textbf{c}, \textbf{f}, \textbf{j}) displaying a reconstruction of filtered diffraction patterns, showing a clear Moiré superlattice. The Moiré superlattice cell was measured to be \SI{0.47}{nm}, \SI{1.06}{nm}, and, \SI{1.69}{nm}; for \textbf{c}, \textbf{f}, and, \textbf{j}. (\textbf{a}) The two layers of material show a rough surface that seems to arise during or after transfer to the TEM grid.}
    \label{fig:moire_overview}
\end{figure}

\subsection{Analysis of in-plane strain}
\begin{figure}[h]
    \centering
    \def\svgwidth{.74\linewidth}
    \import{resources/Figures}{strain_overview.pdf_tex}
    \caption{Overview of the global strain present in the heterostructure}
    \label{fig:strain_overview}
\end{figure}

