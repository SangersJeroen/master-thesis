\chapter{Introduction}
% \begin{enumerate}

% \item Highlight the significance of working at the nanoscale in terms of technological advancements.

% \item Emphasise that this thesis represents a significant innovation. Mention the unique focus on the integration of 4D-STEM and EMPAD detector.

% \item Clearly state the objectives of the research. Explain the scope, which includes mapping strain, electrostatic potentials, and electric fields in van der waals heterostructures.

% \item Discuss why van der Waals heterostructures is crucial. Emphasise the potential impact on applications. 


% \item Brief outline the methodologies that will be employed, including 4D-STEM

% \item Highlight the contributions of this thesis, such as the development of code for interpreting COM.

% \item Provide an overview of how the thesis is organised, mentioning the subsequent chapters and their roles.
    
% \end{enumerate}


Two-dimensional layered materials have garnered significant attention because they exhibit exciting optical, thermal, and electronic properties that change dramatically as the material is thinned down from bulk to monolayer. The already vast amount of tunability increases even further as one starts to stack these thin two-dimensional building blocks into larger potentially twisted heterostructures. Van der Waals forces that normally hold the layers in the bulk crystal also form the glue that holds these new heterostructures together. This relative ease in manufacturing and the potential applications in fields ranging from power electronics and solar cells to single-photon emitters and biosensors make studying these materials more than worthwhile~\cite{LI2016322, https://doi.org/10.1002/smll.202107059}.\\
Studying these fascinating materials and the dynamical interplay between tunable physical properties such as strain, twist angle, and material combination calls for highly sophisticated and specialised characterisation techniques. Therefore, scanning transmission electron microscopy will be used in combination with a state-of-the-art, fast, and high dynamic range direct electron detector to collect highly localised information. The electron microscope pixelated array detector will be used to capture the diffraction information of a small area in the heterostructure crystal lattice by scanning over the sample with a small localised electron probe in a rasterised pattern to form a four-dimensional dataset. From these four-dimensional datasets, accurate strain, electrical and potential fields can be mapped with micrometre and nanometre precision, respectively. To this end a new to us centre-of-mass analysis approach will be developed and implemented in an existing dataset exploration framework used for rapid experimentation of electron microscope parameters at the time of acquisition.\\
In the report, relevant theory on electron microscopy and crystallography will be presented first in Chapter \ref{sec:theory} after which the fabrication of moiré heterostructures will be explained in Chapter \ref{sec:methods}. How the EMPAD sensor allows us to perform heterostrain and centre-of-mass analysis will be explained in Chapters 4 and 5 respectively. Finally, the conclusions and future possibilities will be discussed in Chapter \ref{sec:outlook}.

