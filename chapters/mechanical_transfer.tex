\section{Fabrication of two-dimensional (2D) Moiré heterostructures}
\label{sec:methods}
%
%Start the chapter with a brief introduction that outlines the importance of fabricating 2D Moire heterostructures and their relevance to your research.

% 2d mat. promising for future applications 
% hughe degree of custimisability with strain- twisttronics and choice of material
% engineering perfect meta-material for application specific needs
% Stamping allows for fast prototyping

The fabrication process of 2D Moir\'e heterostructures plays a pivotal role in determining the properties and performance of these structures. In this thesis, we refer to this process as `mechanical transfer', where two monolayers are stamped together to form a bilayer heterostructure. This step significantly influences the properties and performance of these structures. 
%
Through the `mechanical transfer' process, we gain control over important parameters such as layer thickness, orientation, and lattice mismatch of individual layers, which are key factors in determining the interference pattern and the overall properties of the heterostructure.

In the following, I will explain the step-by-step process we followed to fabricate these structures. This includes exfoliating single layers, verifying their thickness, assembling them, and transferring them to a specialised microscope.


\subsection{Exfoliation of monolayer material}
%
For the preparation of the heterostructures, monolayer flakes were prepared using a three-step process.
%
First, the bulk crystal material was exfoliated two to three times between Scotch tape, using the same method commonly used for graphene~\cite{novoselovRoomTemperatureQuantumHall2007}. 
%
These flakes were then transferred to a polydimethylsiloxane (PDMS) stamp.
%
This was done by placing the PDMS stamp on the Scotch tape with the flakes and then quickly peeling the stamp off with tweezers. 
%
It is important to note that for PDMS, the adhesion force to the flake is proportional to the peel-off speed~\cite{kusakaMicrocontactPatterningConductive2015}.
%
Third, the flakes present on the PDMS stamp were inspected using an optical microscope before being exfoliated using PDMS stamps. At this stage, PDMS stamps are used for exfoliation, as they provide a gentler way of further cleaving the bulk flakes and eliminating a further transfer step from tape to PDMS if a monolayer flake were to be exfoliated using tape since stamping on a substrate with tape is difficult. The exfoliation using two PDMS stamps is performed by laying the stamps on another while on a glass slide to provide support, after which both stamps are peeled from the glass before separation by two tweezers. Both stamps are now inspected by means of an optical microscope, and if monolayer (ML) material is present on either of the stamps it can later be used in creating a heterostructure.
If no ML material is present on a stamp, the third step can be repeated until there is ML material or the flakes have broken up into unusable small pieces.
Furthermore, sacrificial or pristine PDMS stamps can be used to either remove unwanted small flakes or transfer large flakes to a newer and cleaner PDMS stamp, and this selective transfer is performed under a microscope such that existing ML material can be avoided as this will likely break if exposed to the force caused by peeling away the sacrificial PDMS stamp.

\subsection{Verification of monolayer material}

Since the bandgap of TMDCs becomes direct in a single layer,the  and most interesting moir/'e effects come into play for monolayer flake heterostructures, it is important to determine the number of layers in the flakes used for heterostructures. 
For the first round of samples, optical inspection with a transmittance light microscope proved to be too crude to determine the layer number for flakes thinner than three layers.

Solving this issue involved setting up a new transmittance- and reflectance-mode microscope with a spectroscope attached. The entire setup was copied from another lab and has been proven capable of differentiating between the number of layers in a flake~\cite{frisendaMicroreflectanceTransmittanceSpectroscopy2017,niuThicknessDependentDifferentialReflectance2018}.

%---------------------------------------------------------------------------------------
\begin{figure}[h!]
	\centering
	\def\svgwidth{1\linewidth}
	\import{resources/Figures}{spectroscope_traces.pdf_tex}
	\caption{Transmittance spectra, recorded using a light spectrometer, show flakes of varying thicknesses. A) Transmittance spectra for 1 to 6 layers of $WSe_2$ exfoliated on a PDMS-coated glass substrate using the method described previously. 
    %
    B) Transmittance spectra of 1 to 3 layers of $MoSe_2$.
    %
    C,D) The flakes used to collect the spectra of $WSe_2$ (A, C) and $MoSe_2$ (B,D), respectively.
    %
    In A), the square, circular, and diamond markers denote the location of the peaks of the \textit{A}-,\textit{C}-, and, \textit{D}-exciton of $WSe_2$, found by fitting a linear combination of Lorentzian peaks. The shaded blue region denotes the photon energy range in which the \textit{B}-exciton peaks are expected but not found by the fitting model.
    %
    Similarly, in B), the square and circular marker denote the location of the \textit{A}- and \textit{B}-exciton peaks of $MoSe_2$.}
    %
	\label{fig:spectroscope_traces}
\end{figure}
%---------------------------------------------------------------------------------------


\begin{minipage}[b]{0.4\textwidth}
    \begin{tabular}{c|cc|}
    \cline{2-3}
    \multicolumn{1}{l|}{}         & \multicolumn{2}{l|}{Photon Energy of  Exciton {[\si{\electronvolt}]}} \\ \hline
    \multicolumn{1}{|c|}{lay. \#} & \multicolumn{1}{c|}{A}                                        & B                                       \\ \hline
    \multicolumn{1}{|c|}{1}       & \multicolumn{1}{c|}{1.54}                                     & 1.79                                    \\ \hline
    \multicolumn{1}{|c|}{2}       & \multicolumn{1}{c|}{1.55}                                     & 1.80                                    \\ \hline
    \multicolumn{1}{|c|}{3}       & \multicolumn{1}{c|}{1.56}                                     & 1.80                                    \\ \hline
\end{tabular}
\captionof{table}{Measurements of the dip positions for the A- and B-exciton in $MoSe_2$ flakes of varying layer thickness}
\label{tab:mose2_measurement_spectra}
\end{minipage}%
\hspace{0.1\textwidth}%
\begin{minipage}[b]{0.4\textwidth}
    \begin{tabular}{c|ccc|}
    \cline{2-4}
    \multicolumn{1}{l|}{}         & \multicolumn{3}{l|}{Photon Energy of  Exciton {[\si{\electronvolt}]}} \\ \hline
    \multicolumn{1}{|c|}{lay. \#} & \multicolumn{1}{c|}{A}                   & \multicolumn{1}{c|}{C}                  & D                  \\ \hline
    \multicolumn{1}{|c|}{1}       & \multicolumn{1}{c|}{1.666}               & \multicolumn{1}{c|}{2.429}              & 2.813              \\ \hline
    \multicolumn{1}{|c|}{2}       & \multicolumn{1}{c|}{1.654}               & \multicolumn{1}{c|}{2.323}              & 2.742              \\ \hline
    \multicolumn{1}{|c|}{3}       & \multicolumn{1}{c|}{1.644}               & \multicolumn{1}{c|}{2.226}              & 2.716              \\ \hline
    \multicolumn{1}{|c|}{4}       & \multicolumn{1}{c|}{1.632}               & \multicolumn{1}{c|}{2.200}              & 2.726              \\ \hline
    \multicolumn{1}{|c|}{5}       & \multicolumn{1}{c|}{1.623}               & \multicolumn{1}{c|}{2.200}              & 2.820              \\ \hline
    \multicolumn{1}{|c|}{6}       & \multicolumn{1}{c|}{1.622}               & \multicolumn{1}{c|}{2.182}              & 2.794              \\ \hline
\end{tabular}
\captionof{table}{Measurements of the dip positions for the A-, C- and D-exciton in light transmittance spectra of $WSe_2$ of varying thickness.}
\label{tab:wse2_measurement_spectra}
\end{minipage}
\vspace{1cm}

Using this setup, we recorded transmittance and reflectance spectra for both $WSe_2$ and $MoSe_2$ flakes with varying layer numbers.



The transmittance spectra are displayed in Figure~\ref{fig:spectroscope_traces}A for $WSe_2$ and Figure~\ref{fig:spectroscope_traces}B for $MoSe_2$. 
%
While both transmittance and reflectance spectra should ideally have used the same halogen bulbs for illumination in both transmission and epi-illumination modes, the reflectance spectra exhibited more noise in the energy range of $\leq \SI{1.6}{\electronvolt}$ due to the lower intensity emitted by the epi-illumination bulb in this region. As a result, we omitted the reflectance spectra from this study.

All spectra were collected using a $100\times$-magnification $0.55NA$ objective, along with a \SI{150}{\micro\meter} thick core glass fiber leading to the spectrometer. 
%
The spectrometer collected and averaged 50 spectra, which were integrated over \SI{500}{\milli\second}. To improve the rejection of stray light, the field aperture of the light was fully closed.

As can be seen in the transmittance spectra, the overall intensity difference is the best indicator for flake thickness beyond 2 layers, whereas the location of the A-exciton is the key differentiator between 1 or 2 layers. 
%
Using the spectroscope and the new microscope has greatly reduced the time required and greatly improved the ease of finding thin material and verifying its thickness.

\subsection{Assembling a Heterostructure}
%
Once two suitable monolayer flakes have been located on different PDMS stamps, they can be assembled to form a heterostructure.
%
The assembly process involves stamping these flakes onto a specially prepared substrate using a dedicated stamping setup.
%
This setup enables precise control over the flake positions and consists of two main components: a substrate stage and a flake stage.

The substrate stage offers three degrees of freedom: two micromanipulators control the substrate's $x$- and $y$-positions under the microscope, and the stage itself can rotate to align the edges of the monolayer flakes to be stamped.
%
The stamping stage can move a glass slide with a stamp along three axes.
%
The complete stamping setup is displayed in Figure \ref{fig:stamping_set-up}, adapted  from previously published setups~\cite{castellanos-gomezDeterministicTransferTwodimensional2014, castellanos-gomezDeterministicTransferTwodimensional2014a}.

%--------------------------------------------------------------------------
\begin{figure}[h]
	\centering
	\def\svgwidth{1\linewidth}
	\import{resources/Figures}{stamping.pdf_tex}
	\caption{Schematic outline of the stamping process: 1) Before strating the stamping process, two PDMS stamps with monolayer material need to be prepared, along with a silicon substrate coated with PVA.
    %
    2) The first flake is stamped, preferably as close to the centre of rotation of the substrate stage as possible, as this placement makes aligning the subsequent flake easier.
    %
    3) After stamping, the micromanipulator with the PDMS stamp is slowly raised, peeling the stamp off the substrate and leaving the first monolayer flake behind. 
    %
    4) The second stamp, carrying another layer of monolayer material, is then aligned by rotating the substrate stage and moving the stamping micromanipulator. 
    %
    5) This stamp is carefully lowered onto the substrate, and any misalignment are corrected as necessary. 
    %
    6) Finally, the second stamp is removed slowly, ensuring both flakes are properly transferred to the substrate.}
    
\label{fig:stamping_process}
\end{figure}
%--------------------------------------------------------------------------

The stamping setup comprises a reflective optical microscope, a rotating substrate stage connected to two micromanipulators, and a stamping stage connected to three micromanipulators.
%
The monolayer material is stamped onto the substrate by first adhering the PDMS stamp onto a glass microscope slide.
%
This slide is then inserted into the holding mechanism on the stamping stage, as illustrated in Figure \ref{fig:stamping_process}.
%
During this process, it is essential to ensure that the flake adheres to the substrate and not the stamp.
%
To achieve this, the speed at which the stamp is peel-off is crucial, it should be as slow as possible without interrupting the process.
%
Additionally, it is of utmost importance to avoid introducing any vibrations or other forces during this step, as they can easily damage or tear the flake.

%--------------------------------------------------------------------------
\begin{figure}[hb]
	\centering
	\def\svgwidth{1\linewidth}
	\import{resources/Figures}{stamping_setup.pdf_tex}
    \caption{Photograph of the stamping setup used for creating heterostructures on a polymer-coated diced silicon wafer. 
    %
    From top to bottom in the setup, the components are as follows: a camera connected to an adjacent computer providing a live view of the process, a variable zoom lens (ranging from $\times$2\--$\times$12) for magnification, a glass fibre for epi-illumination (halogen source not pictured), a piece of carbon tape used to secure the diced silicon chip, micromanipulators for positioning the silicon substrate and potential flake, a glass slide with a PDMS stamp and flake, and additional micromanipulators used for precise stamp positioning.}
	\label{fig:stamping_set-up}
\end{figure}
%--------------------------------------------------------------------------

Applying the second stamp with monolayer material requires extra care and precision because  it must be aligned with the first already stamped monolayer flake. 
%
The alignment process involves locating the monolayer material on the substrate, roughly aligning the micromanipulator with the second flake, and then aligning the edges of both flakes to the desired angle.
%
Changing the angle of the flake on the substrate requires rotating the stage, which may likely move the stamped flake out of view. 
%
After alignment of the angles, it becomes possible to position the flakes on top of each other using the micromanipulators. 
%
Slowly lowering the second stamp to the substrate will bring both flakes into view, allowing for a more precise alignment.
%
During the further lowering of the flake, its best to zoom out to see where the PDMS stamp first contacts the substrate. 
%
If this initial contact point is too far from the ML material, further lowering of the stamp can squeeze the PDMS and cause the flake to shift in the direction of the contact-front propagation during further lowering.
%
If this happens, it is possible to restart the stamping procedure and compensate for the movement, but it is advised to remove the glass slide with the flake completely and cut the PDMS in a way that the initial contact is closer to the monolayer material.

After stamping of the stamping of the second flake, the PDMS needs to be slowly peeled off again in the same manner as before.
%
Afterward, the result will be a heterostructure on the prepared substrate, ready for transfer to a TEM grid.

\subsection{Transferring to a TEM grid}

Transfer of the heterostructure to a TEM grid was performed using a previously devised polymer assisted method~\cite{kosterPolymerassistedTEMSpecimen2021}, in which a diced silicon wafer is coated with a polymer prior to stamping.
%
The flakes are then stamped directly onto the polymer coating in the same manner as described in the previous section. Afterward, they are covered with a TEM grid. The application of a single drop of either water or IPA,  depending on the polymer used, causes the flexible holey carbon film of the TEM grid to connect with the polymer coating by means of surface tension as the drop evaporates. This process "seals" the heterostructure between the holey-carbon film and the polymer.

By adding another drop, this time the solvent for the polymer, the flake held by the TEM grid and the grid itself can separate from the silicon substrate, which then allows it to be picked up using tweezers.

The process is illustrated in Figure \ref{fig:wet_transfer}.

%------------------------------------------------------------------------------
\begin{figure}[h]
	\centering
	\def\svgwidth{1\linewidth}
	\import{resources/Figures}{wet_transfer.pdf_tex}
	\caption{Schematic illustration of the transfer process: 1) A previously prepared heterostructure is located on the substrate, and a suitable TEM grid is carefully placed over the heterostructures using tweezers. 2) The TEM grid is held in place, for the samples transferred in this work, it was secured by pressing a microscope slide onto the rim of the TEM grid using a stamping micromanipulator, taking care not to exert excessive pressure, which could push the TEM grid into the PVA sand prevent IPA from penetrating the interior. 3) After the grid is wetted with a drop of IPA, the resulting surface tension clings to the carbon film, pulling it onto the PVA-coated substrate and heterostructure as the drop evaporates. 4) Once all the IPA has evaporated and the glass slide is removed, a drop of distilled water is added to dissolve the PVA. This releases the TEM grid, allowing it to float on the water droplet and be ready to be picked up with tweezers.}
	\label{fig:wet_transfer}
\end{figure}
%------------------------------------------------------------------------------

The authors of the previously cited paper demonstrated that the use of both PMMA and PVA is possible, but they require different solvents: acetone for PMMA and water for PVA, respectively.

For this work, the heterostructures were stamped on a diced silicon wafer coated with \(\sim \) \SI{100}{\nm} thick PVA and were then annealed at \SI{400}{\degreeCelsius} in a vacuum due to significant contamination.


\subsection{Analysis of moiré lattices in stamped $MoSe_2$/$WSe_2$ flakes}
% \section{Characterisation and Strain Analysis of $MoSe_2$/$WSe_2$ Moiré Heterostructures via 4D-STEM EMPAD}
\label{sec:results}

%Missing: A Figure that should include the EMPAD image, a representative EWPC pattern, and the weighted point cloud. Indicate in the text the clusters used for the strain claculation.

%TIP: This section is similar to what we recently published. Indicate in the text that  this section is also part of a paper we recently published. Add the reference (you are co-author).

In the following section, we detail the characterisation of $MoSe_2$/$WSe_2$ moiré heterostructures, serving as a proof of concept for evaluating strain by means 4D-STEM EMPAD.

%---------------------------------------------------------------------------------------
\begin{figure}[p]
    \thisfloatpagestyle{plainlower}
    \centering
    \def\svgwidth{.9\linewidth}
    \import{resources/Figures}{moire_flakes.pdf_tex}
    \caption{(\textbf{a}, \textbf{d}, \textbf{h}) HR-TEM images of three different $MoSe_2$/$WSe_2$-Moiré heterostructures stamped under different angles verified by the FFT of images (\textbf{b}, \textbf{e}, \textbf{i}) taken at $460k\times$ magnification (not shown). The Moiré angle can be determined by measuring the angle between the two diffraction peaks in a set, for \textbf{b}, \textbf{e}, \textbf{i} the angle was measured to be $16.6\degree$, $12.2\degree$, and $8.3\degree$, respectively. (\textbf{c}, \textbf{f}, \textbf{j}) displaying a reconstruction of filtered diffraction patterns, showing a clear Moiré superlattice. The Moiré superlattice cell was measured to be \SI{0.47}{nm}, \SI{1.06}{nm}, and, \SI{1.69}{nm}; for \textbf{c}, \textbf{f}, and, \textbf{j}. (\textbf{a}) The two layers of material show a rough surface that appears to arise during or after transfer to the TEM grid.}
    \label{fig:moire_overview}
\end{figure}
%---------------------------------------------------------------------------------------


Figure~\ref{fig:moire_overview}(a,d,g) showcases low-magnification TEM images of $MoSe_2$/$WSe_2$ heterostructures along with their corresponding FFTs and inverse Fourier transformed regions.
%
These heterostructures were fabricated using the method outlined in Section~\ref{sec:methods}. 
%
%The flakes were selected for stamping only by utilising an optical microscope to identify thin regions. 

As illustrated in Figure \ref{fig:moire_overview}\textbf{a,d,g}, the flakes are not uniformly thin, and thus, not all consist of two monolayer layers.
%
Nonetheless, the high-resolution TEM images demonstrate that the stamping method can successfully create heterostructures composed of monolayer materials. 
%
Figures \ref{fig:moire_overview}\textbf{b,c,f} display the Fast Fourier Transforms (FFTs) of the HRTEM images. The FFTs displays twelve spots. This indicates the presence of two rotated crystals within the heterostructure.
%
By masking these diffraction peaks and performing the inverse Fourier transform, a clear, noise-free representation of the resulting moiré lattice is obtained (as shown in Figure~\ref{fig:moire_overview}\textbf{c,f,j}). This representation confirms the expected correlation: as the moiré angle decreases, the unit cell size increases. 

