\chapter{Introduction}
%
In recent years, two-dimensional (2D) layered materials \textcolor{red}{add Refs.} have garnered significant attention because they exhibit exciting optical~\textcolor{red}{add Refs.}, thermal~\textcolor{red}{add Refs.}, and electronic~\textcolor{red}{add Refs.} properties that change dramatically as the material is thinned down from bulk to monolayer.
%
The already vast amount of tunability increases even further as one starts to stack these thin 2D building blocks into larger, potentially twisted heterostructures, exploiting Van der Waals forces that normally hold the layers in the bulk crystal~\textcolor{red}{add Refs.}.\\

The relative ease of manufacturing~\textcolor{red}{add Refs.} these heterostrutcures and their potential applications across diverse fields, including power electronics~\textcolor{red}{add Refs.}, solar cells,  single-photon emitters~\textcolor{red}{add Refs.}, and biosensors~\textcolor{red}{add Refs.}, underscore the significance of studying these materials~\cite{LI2016322, https://doi.org/10.1002/smll.202107059}.\\ \textcolor{red}{what do you mean by ease ... is the stamping right? perhaps you could explain it here...thanks to the weak vdW we can easily exfoliate them and stack many different materials one on top of each other creating a wide variety of heterostructures each of them with different .}\\

This research is motivated by the need to understand these fascinating materials and the dynamic interplay between tunable physical properties, such as strain, twist angle (referring to the relative rotation of individual layers in the heterostructure), and material combinations. The manipulation of these properties not only holds promise for novel applications across various fields but also leads to intriguing physical phenomena arising from the rotation of layers in these heterostructures.\\

For instance, \textcolor{red}{...add concrete examples and explain the physics underneath of this configuration. Some examples are: unconventional superconductivity in twisted bilayer graphene; you could also add examples that exhibit SPE, etc...Here also add Refs.}\\

To achieve this goal, we require advanced techniques capable of pinpointing nanometer-scale features, enabling correlations between spatial atomic arrangements and features such as potential fields. Therefore, this study employs scanning transmission electron microscopy (STEM) in combination with a state-of-the-art, high-speed, and high dynamic range direct electron detector to collect precise and localised information. In particular, the electron microscope pixelated array detector (EMPAD) will be used to capture the diffraction data from small areas of the heterostructure crystal lattice. By scanning the sample with a finely localised electron probe in a rasterised pattern, we generate comprehensive four-dimensional (4D) datasets. These 4D datasets enable the mapping of strain, electrical properties, and potential fields with micrometre and nanometre-level precision.\\

To facilitate this researcher, I developed and integrated a novel centre-of-mass (COM) analysis approach into an existing dataset exploration framework. This integration streamlines experimentation during electron microscope data acquisition and subsequent data analysis. \textcolor{red}{Here your main results... So, thanks to your approach you manage to ....}\\

The report begins with a presentation of relevant theory on electron microscopy and crystallography (Chapter \ref{sec:theory}). We then delve into the fabrication of moir\'e heterostructures (Chapter \ref{sec:methods}) and the capabilities of the EMPAD sensor for heterostrain and COM analysis (Chapters \ref{sec:heterostrain} and \ref{sec:COM}, respectively). Finally, we discuss conclusions and future possibilities in Chapter \ref{sec:outlook}.

