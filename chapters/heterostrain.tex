\chapter{Analysis of heterostrain in twisted a \ce{MoSe_2}/\-\ce{WSe_2} heterobilayer structure}
\label{sec:heterostrain}
% State of the art
%
Strain analysis in electron microscopy can be achieved through a multitude of ways, the main distinction between all of them is whether they are performed by analysing a real-space image captured by high-resolution TEM techniques or performed by analysing reciprocal space images captured through scanning-TEM techniques. 
%
An example of the first method is Geometric Phase Analysis (GPA) where displacements of atoms in a crystal are directly observed in the high-resolution image and strain can be computed \cite{HYTCH1998131, hytchGEOMETRICPHASEANALYSIS1997, nguyenAtomicDefectsDoping2017}. 
%
The second method relies primarily on the fact that a CBED/NBED pattern for a small enough convergent electron probe directly measures the local crystal structure. 
%
The reciprocal-space unit cell of the local crystal structure and thus the positions of the peak in the CBED pattern are directly correlated with the size of the real-space unit cell such that a compressing force in the real-space unit cell will elongate the reciprocal-space unit cell in the same direction and vice-versa for a tensile strain \cite{ophusFourDimensionalScanningTransmission2019, vanwinkleRotationalDilationalReconstruction2023, kazmierczakStrainFieldsTwisted2021, hanStrainMappingTwoDimensional2018}, tracking the peaks then gives access to the strain information.  In the following sections the second method will be applied for both large and small field-of-views.

\blfootnote{Parts of this chapter have been published in the journal \textit{Advanced Functional Materials} (2023) by S. E van Heijst, M. Bolhuis, A. Brokkelkamp, J. J. M. Sangers, and S. Conesa-Boj. Additionally, part of the following content is included as part of the Supplementary Information for the manuscript ``Nanoscale Strain Mapping in van der Waals Materials from 4D-STEM: Beyond the Planar Configuration'' by M. Bolhuis, S. E. van Heijst, J. J. M. Sangers, and S. Conesa-Boj, which is currently \textit{under revision} (2023).}

\section{Micrometre field-of-view strain analysis}
%

Strain analysis at the micrometre-scale hinges on capturing clear convergent- or nano-beam electron diffraction patterns as the locations of the peaks therein hold the valuable crystallographic information and the by strain induced deviation thereof, thickness and local sample tilt affect the intensity distributions of the patterns in such a way that tracking the peaks over micrometre distances is challenging. % SC show this with an image
%
The average distance between folds, wrinkles, or, the sides of holes in the lacy carbon grids is typically smaller than a micrometre-scale field-of-view; these features tilt the crystalline sample such that the optical axis of the electron microscope is no longer parallel to the crystal zone-axis. Overcoming the effects of local tilt, wrinkles, folds, and, multilayeredness found in stamped heterostructures is achieved by implementing a power cepstrum transform before peak tracking.
%
The power cepstrum transform (equation \ref{eq:cepstrum}) was originally developed for speech analysis \cite{1570854175999207936, oppenheimDspHistoryFrequency2004,nollCepstrumPitchDetermination1967} and later adapted for application to the post-specimen electron exit-wave ($\psi(\mathbf{q})$) whose magnitude is directly probed by the EMPAD \cite{padgettExitwavePowercepstrumTransform2020} as this is the CBED/NBED pattern.

%-------------------------------------------------------
\begin{equation}
	PC\{f(t)\} = \left| \mathscr{F} \left\{ \ln{\left| \mathscr{F}\{f(t)\} \right|^2} \right\} \right|^2
	\label{eq:cepstrum}
\end{equation}
%-------------------------------------------------------

The intensity of the recorded pattern can be described as a convolution between the image of the electron beam probe function in momentum-space ($\Phi(\mathbf{q})$) convoluted with the by the envelope function ($E(\mathbf{q})$) multiplied true object function ($\mathcal{O}(\mathbf{q})$) (equation \ref{eq:cbed_comp}); the envelope function encompasses the intensity attenuation due to local sample tilt and thickness, the true tilt-free object function is a sum of delta-peaks describing the crystal structure in momentum space; such that the result is an attenuated pattern formed by placing an image of the electron-beam probe on each delta-peak. 

%-------------------------------------------------------
\begin{alignat}{2}
	I(\vec{q}) &  & = \left| \Phi(\vec{q}) \otimes \left( E(\vec{q}) \cdot \mathcal{O}(\vec{q}) \right) \right|^2 & = \left| \psi(\vec{q}) \right|^2
	\label{eq:cbed_comp}                                                                                                                             \\
	\ArrowBetweenLines*[\downarrow^1\hspace{1cm}]%
	I(\vec{q}) &  & \approx \left|E(\vec{q}) \right|^2 \left|\Phi(\vec{q}) \otimes \mathcal{O}(\vec{q}) \right|^2 &
	\label{eq:cbed_approx}
\end{alignat}
%-------------------------------------------------------

As can be seen in Figure \ref{fig:adf_nbed_ewpc}b where the intensity of the disks is suppressed towards the edges of the pattern. Under the ideal conditions used for strain-analysis the probe-image is very small, and the envelope function varies little over the whole pattern resulting in a relatively unvarying offset. Therefore; assumption $\downarrow^1$ states that multiplication by the envelope function after convolution is approximately similar to convolution of the multiplied result, such that the intensity distribution can be rewritten to the form of equation \ref{eq:cbed_approx}. Using this simplification, the definition of the power-cepstrum, and, the knowledge that the CBED/NBED patterns are a mechanically computed Fourier transform of the real-space intensity distribution; the exit-wave power-cepstrum of the CBED/NBED pattern can be written as in equation \ref{eq:cepstrum_unsimp}.

\begin{alignat}{3}
	EWPC_{\psi(\vec{q})} & =\left| \mathscr{F}\left\{\ln{\left| E(\vec{q}) \right|^2}\right.\right. &  & +\left. \left.\ln{\left| \Phi(\vec{q}) \otimes \mathcal{O}(\vec{q}) \right|^2}\right\} \right|^2 & \label{eq:cepstrum_unsimp} \\
	\ArrowBetweenLines*[\downarrow^2\hspace{-2cm}]%
	                     &                                                                          &  & \approx \Phi(\vec{q}) \otimes \ln{\left| \mathcal{O}(\vec{q}) \right|^2}                         &                            \\
	                     & \approx PC\left\{ \epsilon(\vec{x})\right\}+                             &  & \left| \phi(\vec{x})\right|^2 \cdot PC\left\{ o(\vec{x})\right\}                                 & \label{eq:cepstrum_simp}
\end{alignat}

Furthermore, if the ronchigram is well-formed and corrected, and a small aperture is used to select only the centre; the second term in equation \ref{eq:cepstrum_unsimp} can be simplified further if the Bragg-spot separation is larger than the convergence angle of the electron-probe, such that the Bragg disks do not overlap. 
%
This assumption $\downarrow^2$ allows for the separation of the EWPC in to a summation of the power-cepstrum of the envelope function and the power-cepstrum of the crystal-structure information \cite{padgettExitwavePowercepstrumTransform2020}. 
%
Since the envelope function in the CBED/NBED pattern varies slowly over the image and the diffraction Bragg-spots are a high frequency repeating pattern both features occupy a different frequency band in the image, they are thus separated by the power-cepstrum transform. The tilt and thickness information is then transformed to and muddles the centre of the EWPC pattern in Figure \ref{fig:adf_nbed_ewpc}c, the crystal-structure information is well separated and occupies a different section in the image.

%-------------------------------------------------------
\begin{figure}
	\centering
	\def\svgwidth{0.9\linewidth}
	\import{resources/Figures}{adf_cbed_ewpc.pdf_tex}
	\caption{\textbf{a}) virtual annular dark-field image reconstructed by masking the dataset such that the detector geometry is equivalent to that of the HAADF-detector. Image is taken at the transition region of monolayer \ce{MoSe_2} (bottom) to a heterostructure of \ce{MoSe_2} and \ce{WSe_2} (top) on a lacy carbon grid. The faint circles in the middle of the image are contamination from inspection in TEM mode. Contamination in STEM mode caused the bright white spots on the left side of the image. \textbf{b}) logarithm of the nanobeam electron-diffraction pattern captured in the heterobilayer region of the image in \textbf{a}. Two sets of six diffraction disks, one set of six for each material, can be seen in the pattern. Due to local tilt of the crystalline layers diffraction disks on one side of the central $000$-disk, in this case the bottom, are brighter than disks on the opposing side. \textbf{c}) Exit-wave power cepstrum (EWPC) transform of the NBED pattern displayed in \textbf{b}. The EWPC pattern does not suffer from unequal intensity distributions. Peaks in the pattern are distinguishable in the EWPC transform.}
	\label{fig:adf_nbed_ewpc}
\end{figure}
%-------------------------------------------------------

Beam conditions optimised for strain analysis at the micrometre-scale need to result in being able to separate the Bragg disk on the EMPAD sensor as well as fitting as many spots within the bright part of the envelope on the sensor by optimising the camera length. Generally a smaller spot size, around 5-6, is used to increase electron brightness and the semi-convergence angle of the electron probe is situated around $\approx$\SI{5}{\milli\radian} to achieve well-defined round disks. 


\section{Analysis of hetero-strain in \ce{MoSe_2}/\ce{WSe_2} heterostructures}

%---------------------------------------------------------------------------------------
\begin{figure}[t]
    \centering
    \def\svgwidth{.95\linewidth}
    \import{resources/Figures}{point_cloud.pdf_tex}
    \caption{\textbf{a}) A nanobeam electron diffraction pattern recorded at a single probe position by the EMPAD, plotted in log scale. Diffraction peaks are located near the centre and intensities modulated by an envelope function. \textbf{b} Plot of the exit-wave power cepstrum transform of the NBED pattern picture in \textbf{a}. \textbf{c} Point cloud denoting the number of occurrences for every peak displayed in \textbf{b} for all probe positions in the dataset. Bottom half shows all the peaks found throughout the dataset, top half shows only peak positions that were found more than 80 times throughout the whole dataset.}
    \label{fig:point_cloud}
\end{figure}
%---------------------------------------------------------------------------------------

For the heterostrain analysis, the heterostructure shown in Figure~\ref{fig:moire_overview}\textbf{d} was selected because its transfer to the TEM grid was problem-free, eliminating any ambiguities in the material. In contrast, for the other two heterostructures depicted in Figure~\ref{fig:moire_overview}, numerous flakes swirled around in the IPA during the dissolution of the PVA substrate. The two layers were stamped on top of each other at an angle of \SI{12.2}{\degree}, forming a moiré lattice with a periodicity of \SI{1.06}{\nano\meter}.

Heterostrain analysis was performed using the exit wave power cepstrum (EWPC) transform method. A single 4D-dataset was captured for the heterostructure, the nano-beam electron diffraction pattern of the exit-wave contains crystallographic information on both layers, such that only a single NBED pattern (Fig.~\ref{fig:point_cloud}\textbf{a}) is needed at every probe position to capture strain information in both layers. After applying the EWPC transform (Fig.~\ref{fig:point_cloud}\textbf{b}) both layers were analysed separately, tracking the corresponding EWPC peaks to only consider the strain and rotation within each single layer. For this, peaks 7 and 10 were used for the top layer that extends for half the image (Fig.~\ref{fig:adf_nbed_ewpc}\textbf{a}), and peaks 4 and 1 were used for the bottom layer. Point clouds are shown, both filtered (top half) and unfiltered (bottom half), in Figure \ref{fig:point_cloud}\textbf{c}. Filtering in this context means rejecting positions where less than 80 peaks have been found.

Figure~\ref{fig:strain_overview} presents the results, with each layer shown in a separate column. From top to bottom, the rows represent compressive/tensile strain in the $x$ and $y$ directions, shear strain, and layer rotation, all relative to the reference area denoted by the red square. 
%
The rotation in the bottom row differs from the moiré angle under which the layers were stamped. White areas indicate regions where the targeted peaks were not detected, thus no information could be retrieved.
%
Figure \ref{fig:strain_overview}\textbf{a-c, e-d} reveals that both layers experience strains ranging between \SI{-5}{\percent} and \SI{5}{\percent}. This strain is primarily localised near the edges of the supporting carbon mesh over which both layers rest. 
%
For the top layer, strain is also evident near a vertical fold in the layer. 
%
The relative rotation plot for the top layer highlights a pronounced difference in relative rotation on either side of the fold.
%
This suggests that the rotation likely pushed the material inward, resulting in the fold. 
%
The relative rotation diminishes as the fold fades when moving away from the flake's edge.

%---------------------------------------------------------------------------------------
\begin{figure}[p]
    \thisfloatpagestyle{plainlower}
    \centering
    \def\svgwidth{.74\linewidth}
    \import{resources/Figures}{strain_overview.pdf_tex}
    \caption{Overview of the heterostrain present in both the top and bottom layer of the heterostructure displayed in Figure \ref{fig:moire_overview}\textbf{d} for the left and right column respectively. The red box denotes the reference region for the strain analysis. In any white regions, not enough peaks were found to perform strain analysis. \textbf{a,b,e,f}) Show compressive or tensile strain in the top layer for \textbf{a, b} and in the bottom layer for \textbf{e, f}. \textbf{c, g}) illustrate the shear strain in both layers. \textbf{e, h} show the local rotation of the layer with respect to the average rotation in the red box.}
    \label{fig:strain_overview}
\end{figure}
%---------------------------------------------------------------------------------------
