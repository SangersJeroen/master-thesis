\section{Fabrication of two-dimensional moiré heterostructures}
\subsection{Mechanical transfer}
Mechanical transfer in this thesis relates to the practise of both the stamping of two monolayers on top of one-another to form a bilayer heterostructure and the transferring of that heterostructure from the substrate it was stamped on to the holey-carbon TEM grid for inspection in the electron microscope.

\subsubsection{Exfoliation of monolayer material}
For the preparation of the heterostructures, monolayer flakes were prepared using a three-step process.
Firstly bulk crystal material was exfoliated between scotch tape two to three times, using the same method as often is used for graphene \cite{novoselovRoomTemperatureQuantumHall2007}, to prepare fresh flakes.
These fresh flakes would then be transferred to a Polydimethylsiloxane (PDMS) stamp by placing the PDMS stamp on the scotch tape with the flakes before quickly peeling the stamp of with tweezers. For this step it is important that the peeling speed is high as for PDMS the adhesion force to the flake is proportional to the peel-off speed \cite{kusakaMicrocontactPatterningConductive2015}.
Thirdly, the flakes present on the PDMS stamp were inspected using an optical microscope before being exfoliated using to PDMS stamps. At this stage PDMS stamps are used for exfoliation as they provide a gentler way of further cleaving the bulk flakes as well as eliminating a further transfer step from tape to PDMS if a monolayer flake were to be exfoliated using tape since stamping on a substrate with tape is difficult. The exfoliation using two PDMS stamp is performed by laying the stamps on another while on a glass slide to provide support after which both stamps are peeled from the glass before separation by two tweezers. Both stamps are now inspected by means of an optical microscope, if monolayer (ML) material is present on either of the stamps it can later be used in creating a heterostructure.
If no ML material is present on a stamp the third step can be repeated until there is ML material or the flakes have broken up into unusable small pieces.
Furthermore, sacrificial or pristine PDMS stamps can be used to either remove unwanted small flakes or transfer large flakes to a newer and cleaner PDMS stamp, this selective transfer is performed under a microscope such that existing ML material can be avoided as this will likely break if exposed to the force caused by peeling away the sacrificial PDMS stamp.

\subsubsection{Assembling a heterostructure}
\begin{figure}[h]
    \centering
    \def\svgwidth{1\linewidth}
    \import{resources/Figures}{stamping.pdf_tex}
    \caption{Overview of the global strain present in the heterostructure}
    \label{fig:stamping_process}
\end{figure}
Once two suitably matching monolayer flakes have been located on different PDMS stamps they can be assembled to form a heterostructure.
The stamping set-up consists of: a reflective optical microscope, a rotating substrate stage attached to two micromanipulators, and, a stamping stage connected to three micromanipulators.
The substrate stage has three degrees of freedom: two micromanipulators control the $x$- and $y$-position of the substrate under the microscope and the stage itself is free to rotate to align the edges of the to be stamped ML flakes.
The stamping stage is capable of precisely moving a glass slide with a stamp along three axes.
The complete stamping set-up is pictured in Figure \ref{fig:stamping_set-up} and is a direct adaptation from previously published set-ups \cite{castellanos-gomezDeterministicTransferTwodimensional2014, castellanos-gomezDeterministicTransferTwodimensional2014a}.
The ML material is stamped onto the substrate by first adhering the PDMS stamp onto a glass microscope slide and inserting this slide into a holding mechanism on the stamping stage, after which the stamp is carefully brought into contact with the substrate.
Since we wish for the flake to adhere to the substrate and not the stamp, we now slowly release the stamp from the substrate.
To finish the heterostructure, the last step is repeated for the second flake after having first aligned the ML material on the stamp with that on the substrate.

\subsubsection{Transferring to a TEM grid}
\begin{figure}[h]
    \centering
    \def\svgwidth{1\linewidth}
    \import{resources/Figures}{wet_transfer.pdf_tex}
    \caption{Overview of the global strain present in the heterostructure}
    \label{fig:wet_transfer}
\end{figure}

\section{Advanced electron microscopy techniques to map moiré physics at the nanoscale; an electron microscope pixelated array detector.}

\subsubsection{Electron microscope pixel-array detector}
%In TEM operating mode a parallel electron beam is used to illuminate the sample and form an image on either the phosphorous screen or digital camera, this spreads the beam over a larger area such that the local electron dose is relatively uniform.
In a scanning-TEM (STEM) mode the beam is focused to a small probe-like point at the specimen. Elastic scattering then deflects the electrons from the optical axis onto one of the ring shaped detectors encircling the optical axis, these detectors are called the dark field detectors. These singular annular detectors bunch the elastically scattered electrons together, losing valuable information on the exact scattering angle. The electron microscope pixel-array detector (EMPAD) solves this problem by replacing the annular dark field and singular bright field detectors with a single fast-readout, high dynamic range pixel grid on which every pixels' electron dose is stored separately such that after acquisition of a complete STEM scan the bright- and dark-field detectors can be virtually recreated by integrating the electron dose using annular or circular masks on the data.
The pixelated nature of the detector enables the precise computation of the intensity distribution of not only the whole CBED patten but also within each diffraction disk within the CBED pattern, greatly improving the potential resolution of ptychography methods \cite{pennycookEfficientPhaseContrast2015, yangEfficientPhaseContrast2015a} as well as enabling charge density analysis \cite{hachtelSubAngstromElectricField2018,wenMapping1DConfined2022,fangAtomicElectrostaticMaps2019}.
A greater accuracy in scattering angle is also available since every pixel in the array has its own smaller range of scattering angles whose electrons the pixel collects.


\subsubsection{Strain analysis}

\subsubsection{Ptychography}

\subsubsection{Charge density analysis}




