\section{Characterisation and Strain Analysis of $MoSe_2$/$WSe_2$ Moiré Heterostructures via 4D-STEM EMPAD}
\label{sec:results}

%Missing: A Figure that should include the EMPAD image, a representative EWPC pattern, and the weighted point cloud. Indicate in the text the clusters used for the strain claculation.

%TIP: This section is similar to what we recently published. Indicate in the text that  this section is also part of a paper we recently published. Add the reference (you are co-author).

In the following section, we detail the characterisation of $MoSe_2$/$WSe_2$ moiré heterostructures, serving as a proof of concept for evaluating strain by means 4D-STEM EMPAD.

%---------------------------------------------------------------------------------------
\begin{figure}[p]
    \thisfloatpagestyle{plainlower}
    \centering
    \def\svgwidth{.9\linewidth}
    \import{resources/Figures}{moire_flakes.pdf_tex}
    \caption{(\textbf{a}, \textbf{d}, \textbf{h}) HR-TEM images of three different $MoSe_2$/$WSe_2$-Moiré heterostructures stamped under different angles verified by the FFT of images (\textbf{b}, \textbf{e}, \textbf{i}) taken at $460k\times$ magnification (not shown). The Moiré angle can be determined by measuring the angle between the two diffraction peaks in a set, for \textbf{b}, \textbf{e}, \textbf{i} the angle was measured to be $16.6\degree$, $12.2\degree$, and $8.3\degree$, respectively. (\textbf{c}, \textbf{f}, \textbf{j}) displaying a reconstruction of filtered diffraction patterns, showing a clear Moiré superlattice. The Moiré superlattice cell was measured to be \SI{0.47}{nm}, \SI{1.06}{nm}, and, \SI{1.69}{nm}; for \textbf{c}, \textbf{f}, and, \textbf{j}. (\textbf{a}) The two layers of material show a rough surface that appears to arise during or after transfer to the TEM grid.}
    \label{fig:moire_overview}
\end{figure}
%---------------------------------------------------------------------------------------

\subsection{Analysis of moiré lattices in stamped $MoSe_2$/$WSe_2$ flakes}

Figure~\ref{fig:moire_overview}(a,d,g) showcases low-magnification TEM images of $MoSe_2$/$WSe_2$ heterostructures along with their corresponding FFTs and inverse Fourier transformed regions.
%
These heterostructures were fabricated using the method outlined in Section~\ref{sec:methods}. 
%
%The flakes were selected for stamping only by utilising an optical microscope to identify thin regions. 

As illustrated in Figure \ref{fig:moire_overview}\textbf{a,d,g}, the flakes are not uniformly thin, and thus, not all consist of two monolayer layers.
%
Nonetheless, the high-resolution TEM images demonstrate that the stamping method can successfully create heterostructures composed of monolayer materials. 
%
Figures \ref{fig:moire_overview}\textbf{b,c,f} display the Fast Fourier Transforms (FFTs) of the HRTEM images. The FFTs displays twelve spots. This indicates the presence of two rotated crystals within the heterostructure.
%
By masking these diffraction peaks and performing the inverse Fourier transform, a clear, noise-free representation of the resulting moiré lattice is obtained (as shown in Figure~\ref{fig:moire_overview}\textbf{c,f,j}). This representation confirms the expected correlation: as the moire angle decreases, the unit cell size increases. 


\subsection{Analysis of hetero-strain in $MoSe_2$/$WSe_2$ heterostructures}

For the heterostrain analysis, the heterostructure shown in Figure~\ref{fig:moire_overview}\textbf{d} was selected because its transfer to the TEM grid was problem-free, eliminating any ambiguities in the material. In contrast, for the other two heterostructures depicted in Figure~\ref{fig:moire_overview}, numerous flakes swirled around in the IPA during the dissolution of the PVA substrate. The two layers were stamped on top of each other at an angle of \SI{12.2}{\degree}, forming a moiré lattice with a periodicity of \SI{1.06}{\nano\meter}.

The heterostrain analysis was carried out using the exit-wave power cepstrum (EWPC) transform method. Both layers were analysed separately, tracking the corresponding EWPC peaks to solely consider the strain and rotation within each single layer. Figure~\ref{fig:strain_overview} presents the results, with each layer shown in a separate column. 
%
From top to bottom, the rows represent compressive/tensile strain in the $x$ and $y$ directions, shear strain, and layer rotation, all relative to the reference area denoted by the red square. 
%
The rotation in the bottom row differs from the moiré angle under which the layers were stamped. White areas indicate regions where the targeted peaks were not detected, thus no information could be retrieved.
%
Figure \ref{fig:strain_overview}\textbf{a-c, e-d} reveals that both layers experience strains ranging between \SI{-5}{\percent} and \SI{5}{\percent}. This strain is primarily localised near the edges of the supporting carbon mesh over which both layers rest. 
%
For the top layer, strain is also evident near a vertical fold in the layer. 
%
The relative rotation plot for the top layer highlights a pronounced difference in relative rotation on either side of the fold.
%
This suggests that the rotation likely pushed the material inward, resulting in the fold. 
%
The relative rotation diminishes as the fold fades when moving away from the flake's edge.

%---------------------------------------------------------------------------------------
\begin{figure}[p]
    \thisfloatpagestyle{plainlower}
    \centering
    \def\svgwidth{.74\linewidth}
    \import{resources/Figures}{strain_overview.pdf_tex}
    \caption{Overview of the heterostrain present in both the top and bottom layer of the heterostructure displayed in Figure \ref{fig:moire_overview}\textbf{d} for the left and right column respectively. The red box denotes the reference region for the strain analysis. In any white regions, not enough peaks were found to perform strain analysis. \textbf{a,b,e,f}) Show compressive or tensile strain in the top layer for \textbf{a, b} and in the bottom layer for \textbf{e, f}. \textbf{c, g}) illustrate the shear strain in both layers. \textbf{e, h} show the local rotation of the layer with respect to the average rotation in the red box.}
    \label{fig:strain_overview}
\end{figure}
%---------------------------------------------------------------------------------------
\newpage


%
\section{Analysis of .... in twisted $WSe_2$ moire superlattice}

%---------------------------------------------------------------------------------------
\begin{figure}
    \centering
    \def\svgwidth{.95\linewidth}
    \import{resources/Figures}{e2_overview.pdf_tex}
    \caption{\textbf{a}) High-angle annular dark field photograph, image shows material increasing in layer number from the bottom left to the top right. These layers were slightly rotated during transfer, leading to the formation of the moiré pattern. Across this material, another thin strip has settled during transfer, causing a further modulation of the moiré period. \textbf{b}) High-resolution TEM image of the region highlighted in \textbf{a} with a moiré unit cell outlined. \textbf{c}) FFT of the high-resolution image showing a \SI{2.3}{\degree} rotation between the strip and the larger layers.}
    \label{fig:dub_moire}
\end{figure}
%---------------------------------------------------------------------------------------

This sample of stamped $WSe_2$ was created using the described mechanical transfer. Upon inspection in STEM mode, the large features highlighted with the green outline in Figure~\ref{fig:dub_moire}\textbf{a} appeared and are the effect of three different layers of material slightly out of alignment with each other. 
%
The resulting moiré unit cell is highlighted with the dashed outline in Figure~\ref{fig:dub_moire}\textbf{b} and is measured to be caused a \SI{2.3}{\degree} rotation between the thin strip's crystal lattice (highlighted in green) and the moiré pattern present in the underlying layers (highlighted in blue).

%Missing a justification what we are looking at "electric field measurements?"

In the following sections, we will examine three distinct regions from our previously discussed sample: i) a singular hole in otherwise uniform material (to validate our centre-of-mass (COM) analysis tools), ii) a pronounced periodicity within the moiré lattice (to gauge anticipated outcomes), and  iii) a region characterised by double misalignment.

\subsection{Validation of COM analysis tools}

%-----------------------------------------------------------------------------------
\begin{figure}
    \centering
    \def\svgwidth{.95\linewidth}
    \import{resources/Figures}{hole_displacement.pdf_tex}
    \caption{Result of performing CBED shift analysis on a large hole in otherwise uniform material. \textbf{a,b,c,d}) Highlight the shift in the y-direction, the shift in the x-direction, the magnitude of the shift, and, the angle of the displacement respectively. All values are given are (sub)pixel shifts across the sensor. The axes correspond with those given in the vHAADF image \textbf{e}.}
    \label{fig:hole_dis}
\end{figure}
%-----------------------------------------------------------------------------------

In the previous theory section on centre-of-mass (COM) analysis, we highlighted two different scenarios, each necessitating a different analysis approach to COM due to varying effects on the diffracted electron beam.
%
One such scenario arises when the electron probe's cross-sectional area is significantly smaller than the feature under investigation, as seen with the hole in the otherwise uniform moiré bilayer depicted in Figure~\ref{fig:hole_dis}\textbf{e}. 
%
As reported in other works, changes in sample geometry, such as edges and slopes, can deflect the electron bundle. 
%
This deflection tends towards the thicker regions in the sample~\cite{ophusFourDimensionalScanningTransmission2019a,dekkers1974differential}. 
%
By monitoring the periphery of the bright field (BF) disk and its variation relative to its average position, we can discern the deflection of the BF disk in both the $x$- and $y$-directions. 
%
This deflection in the $x$ and $y$ direction is shown in Figure~\ref{fig:hole_dis}\textbf{a,b}, while the combined deflection magnitude and angle are illustrated in Figure~\ref{fig:hole_dis}\textbf{c,d} respectively. 
%
The observed phenomena align with expectations, with the electron beam displacements moving towards the material. 
%
Interestingly, several probe points in the hole's center show minimal displacement, indicating that the beam's cross-sectional area is indeed smaller than the hole itself.
%
Leveraging the data on  BF disk displacement, we can also calculate the momentum transfer conferred to the electron beam, using similar methods as described in~\cite{mullerAtomicElectricFields2014}. 
%
The net momentum transfer and its direction are plotted in Figure \ref{fig:hole_mom}, accompanied by a virtual-HAADF image for context. 
%
The plot shows higher momenta transfer at probe positions nearer to the hole's edge, which diminishes as the probe moves further across the sample or nearer to the hole's center.

\begin{figure}
    \centering
    \def\svgwidth{.7\linewidth}
    \import{resources/Figures}{hole_momentum_transfered.pdf_tex}
    \caption{Left panel show a virtual-HAADF image taken by masking the dataset. Right panel shows the momentum transfer imparted on the electron beam by the sample, the electron beam gets deflected towards the thicker material.}
    \label{fig:hole_mom}
\end{figure}

\subsection{Mapping electric field in a Sub-\SI{1}{\degree} $WSe_2$ Moiré superlattice}

Data pertaining to the moiré lattice was collected from the region highlighted in blue in Figure~\ref{fig:moire_overview}.
%
The moiré angle is estimated to be \SI{\leq 1}{\degree}. Analogous to the hole dataset, we calculated the shift in the BF disk for probe positions and subsequently made corrections in the dataset to align all the BF disks appropriately. This shift in the BF disk is plotted in Figure~\ref{fig:m_dis}. 

%--------------------------------------------------------------------------------------
\begin{figure}[h]
    \centering
    \def\svgwidth{.95\linewidth}
    \import{resources/Figures}{moire_displacement.pdf_tex}
    \caption{Result of performing CBED shift analysis on the region of a moiré lattice highlighted in blue in Figure \ref{fig:dub_moire}. \textbf{a,b,c,d}) Highlight the shift in the y-direction, the shift in the x-direction, the magnitude of the shift, and, the angle of the displacement respectively. All values are given are (sub)pixel shifts across the sensor. The axes correspond with those given in the vHAADF image \textbf{e}.}
    \label{fig:m_dis}
\end{figure}
%--------------------------------------------------------------------------------------

A clear distinction between the unit cell's interior and its walls is evident, as they appear to displace the electron beam diffraction pattern in opposing directions.
%
Upon aligning all the BF disks, we can calculate the COM based solely on the intensity distribution within these disks.
%
The results of this computation are presented in Figure~\ref{fig:m_mom}. 
%
This figure showcases, from top to bottom: a virtual-HAADF image created by masking the dataset, the magnitude and direction of the momentum transfer within the sample plane, and the distribution of charges responsible for this momentum transfer. 
%
Intriguingly, despite showcasing two moiré unit cells, no recurrent features within the COM analysis of the moiré unit cell are discernible.

%--------------------------------------------------------------------------------------
\begin{figure}
    \centering
    \def\svgwidth{.5\linewidth}
    \import{resources/Figures}{moire_efield_charge.pdf_tex}
    \caption{}
    \label{fig:m_mom}
\end{figure}

\subsection{Mapping electric field in a Sub-\SI{2.3}{\degree} $WSe_2$ Moiré superlattice}
\label{sec:large_moire}
% Vortices at the vertices
%--------------------------------------------------------------------------------------
%--------------------------------------------------------------------------------------
\begin{figure}
    \centering
    \def\svgwidth{.95\linewidth}
    \import{resources/Figures}{trip_moire_displacement.pdf_tex}
    \caption{Results from the centre-of-mass analysis performed bright field disk of the double moiré region highlighted in Figure \ref{fig:dub_moire} after the electron diffraction pattern shift was accounted for. \textbf{a,b,c,d}) Highlight the shift in the y-direction, the shift in the x-direction, the magnitude of the shift, and, the angle of the displacement respectively. All values are given are (sub)pixel shifts across the sensor. The axes correspond with those given in the vHAADF image \textbf{e}.}
    \label{fig:trip_m_dis}
\end{figure}
%--------------------------------------------------------------------------------------

In the green-highlighted region of Figure~\ref{fig:moire_overview}\textbf{a}, a double moiré pattern emerges from three layers showcasing two distinct moiré angles.
%
One is near zero degrees, while the other is roughly \SI{2.3}{\degree}.
%
This forms a moiré unit cell about \SI{10}{\nano\meter} in size. 
%
From the virtual-HAADF image derived from this dataset, the probe's cross-sectional area approximates half the size of a moiré unit cell. 
%
Given this, we adjusted for the CBED shift and executed a COM analysis on the BF disk, aiming to delve deeper into the moiré unit cell's structure. 
%
The CBED shift analysis outcomes are displayed in Appendix \ref{appendix:a}.

The subsequent COM analysis, targeting intensity variations due to electric and/or magnetic field, is depicted in Figure~\ref{fig:trip_m_dis}. 
%
One can observe a discernible periodicity in all displacement measurements that aligns with the nuances in the virtual-HAADF image, and a congruence in values across moiré unit cells. 
%
The magnitude of the COM displacement is showcased in Figure~\ref{fig:trip_m_dis}\textbf{c}.
%
Here, a clear distinction between the high-symmetry points, which manifest as white blobs in the virtual-HAADF image (see Figure~\ref{fig:trip_m_dis}\textbf{e}), and the thin semi-disordered regions that connect them.
%
Notably, the COM magnitude redistribution intensifies as one  moves radially from a high-symmetry region's center. 
%
The calculated momentum transfer, shown in Figure \ref{fig:dub_moire}, highlights this effect even more. 
%
For clarity,  the moiré unit cell's outline has been sketched, encircling two high symmetry points, with a third at the vertex. 
%
Interestingly, the two enclosed vortices have a greater magnitude than the vertex vortex, even if their intensity in the virtual-HAADF image don't match. 
%
This observation remains not fully grasped at present and calls for deeper analysis, and theoretical calculations.

%--------------------------------------------------------------------------------------
\begin{figure}
    \centering
    \def\svgwidth{.9\linewidth}
    \import{resources/Figures}{trip_moire_momentum.pdf_tex}
    \caption{Image shows the momentum transfer imparted on the electron beam by features in the sample. The moiré unit cell is highlighted and the same as in Figure \ref{fig:dub_moire}\textbf{b}.}
    \label{fig:trip_m_mom}
\end{figure}
%--------------------------------------------------------------------------------------
