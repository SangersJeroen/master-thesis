\section{Fabrication of two-dimensional moiré heterostructures}
\subsection{Mechanical transfer}
Mechanical transfer in this thesis relates to the practise of both the stamping of two monolayers on top of one-another to form a bilayer heterostructure and the transferring of that heterostructure from the substrate it was stamped on to the holey-carbon TEM grid for inspection in the electron microscope.

\subsubsection{Exfoliation of monolayer material}
For the preparation of the heterostructures, monolayer flakes were prepared using a three-step process.
Firstly bulk crystal material was exfoliated between scotch tape two to three times, using the same method as often is used for graphene \cite{novoselovRoomTemperatureQuantumHall2007}, to prepare fresh flakes.
These fresh flakes would then be transferred to a Polydimethylsiloxane (PDMS) stamp by placing the PDMS stamp on the scotch tape with the flakes before quickly peeling the stamp of with tweezers. For this step it is important that the peeling speed is high as for PDMS the adhesion force to the flake is proportional to the peel-off speed \cite{kusakaMicrocontactPatterningConductive2015}.
Thirdly, the flakes present on the PDMS stamp were inspected using an optical microscope before being exfoliated using to PDMS stamps. At this stage PDMS stamps are used for exfoliation as they provide a gentler way of further cleaving the bulk flakes as well as eliminating a further transfer step from tape to PDMS if a monolayer flake were to be exfoliated using tape since stamping on a substrate with tape is difficult. The exfoliation using two PDMS stamp is performed by laying the stamps on another while on a glass slide to provide support after which both stamps are peeled from the glass before separation by two tweezers. Both stamps are now inspected by means of an optical microscope, if monolayer (ML) material is present on either of the stamps it can later be used in creating a heterostructure.
If no ML material is present on a stamp the third step can be repeated until there is ML material or the flakes have broken up into unusable small pieces.
Furthermore, sacrificial or pristine PDMS stamps can be used to either remove unwanted small flakes or transfer large flakes to a newer and cleaner PDMS stamp, this selective transfer is performed under a microscope such that existing ML material can be avoided as this will likely break if exposed to the force caused by peeling away the sacrificial PDMS stamp.

\subsubsection{Verification of monolayer material}
\begin{figure}[h]
	\centering
	\def\svgwidth{1\linewidth}
	\import{resources/Figures}{spectroscope_traces.pdf_tex}
	\caption{Transmittance spectra, recorded using a light spectrometer, of flakes of varying thicknesses. A) Transmittance spectra for 1 to 6 layers of $WSe_2$ exfoliated, using the previously described method, onto a PDMS on glass substrate. B) transmittance spectra of 1 to 3 layers of $MoSe_2$. C,D) The flakes used for the collecting the spectra of $WSe_2$ (A, C) and $MoSe_2$ (B,D) respectively.
    In A) the square, circular, and, diamond marker; denote the location of the peaks of the \textit{a}-,\textit{c}-, and, \textit{d}-exciton of $WSe_2$ found by fitting to a linear combination of Lorentzian peaks. The shaded blue region denotes the photon energy range in which the \textit{b}-exciton peaks is expected but not found by the fitting model.
Similarly, in B) the square and circular marker denote the location of the \textit{a}- and \textit{b}-exciton peaks of $MoSe_2$.}
	\label{fig:spectroscope_traces}
\end{figure}

\begin{minipage}[b]{0.4\textwidth}
    \begin{tabular}{c|cc|}
    \cline{2-3}
    \multicolumn{1}{l|}{}         & \multicolumn{2}{l|}{Photon Energy of  Exciton {[\si{\electronvolt}]}} \\ \hline
    \multicolumn{1}{|c|}{lay. \#} & \multicolumn{1}{c|}{A}                                        & B                                       \\ \hline
    \multicolumn{1}{|c|}{1}       & \multicolumn{1}{c|}{1.54}                                     & 1.79                                    \\ \hline
    \multicolumn{1}{|c|}{2}       & \multicolumn{1}{c|}{1.55}                                     & 1.80                                    \\ \hline
    \multicolumn{1}{|c|}{3}       & \multicolumn{1}{c|}{1.56}                                     & 1.80                                    \\ \hline
\end{tabular}
\captionof{table}{Measurements of the dip positions for the A- and B-exciton in $MoSe_2$ flakes of varying layer thickness}
\label{tab:mose2_measurement_spectra}
\end{minipage}%
\hspace{0.1\textwidth}%
\begin{minipage}[b]{0.4\textwidth}
    \begin{tabular}{c|ccc|}
    \cline{2-4}
    \multicolumn{1}{l|}{}         & \multicolumn{3}{l|}{Photon Energy of  Exciton {[\si{\electronvolt}]}} \\ \hline
    \multicolumn{1}{|c|}{lay. \#} & \multicolumn{1}{c|}{A}                   & \multicolumn{1}{c|}{C}                  & D                  \\ \hline
    \multicolumn{1}{|c|}{1}       & \multicolumn{1}{c|}{1.666}               & \multicolumn{1}{c|}{2.429}              & 2.813              \\ \hline
    \multicolumn{1}{|c|}{2}       & \multicolumn{1}{c|}{1.654}               & \multicolumn{1}{c|}{2.323}              & 2.742              \\ \hline
    \multicolumn{1}{|c|}{3}       & \multicolumn{1}{c|}{1.644}               & \multicolumn{1}{c|}{2.226}              & 2.716              \\ \hline
    \multicolumn{1}{|c|}{4}       & \multicolumn{1}{c|}{1.632}               & \multicolumn{1}{c|}{2.200}              & 2.726              \\ \hline
    \multicolumn{1}{|c|}{5}       & \multicolumn{1}{c|}{1.623}               & \multicolumn{1}{c|}{2.200}              & 2.820              \\ \hline
    \multicolumn{1}{|c|}{6}       & \multicolumn{1}{c|}{1.622}               & \multicolumn{1}{c|}{2.182}              & 2.794              \\ \hline
\end{tabular}
\captionof{table}{Measurements of the dip positions for the A-, C- and D-excition in light transmittance spectra of $WSe_2$ of varying thickness.}
\label{tab:wse2_measurement_spectra}
\end{minipage}
\vspace{1cm}


Since the bandgap of TMDCs becomes direct in a single layer and most interesting moiré effects come into play for monolayer flake heterostructures its is important to determine the number of layers in the flakes that are used for heterostructures. For the first round of samples, optical inspection with a transmittance light microscope proved to be too crude to determine the layer number for flakes thinner than three layers.
Solving this issue included setting up a new transmittance- and reflectance-mode microscope with a spectroscope attached, the whole set-up is copied from another lab and proven to be able to differentiate between amount of layers in a flake. \cite{frisendaMicroreflectanceTransmittanceSpectroscopy2017,niuThicknessDependentDifferentialReflectance2018}
Using this set-up transmittance- and reflectance-spectra were recorded for both $WSe_2$ and $MoSe_2$ flakes for varying layer numbers, the transmittance spectra are displayed in Figure \ref{fig:spectroscope_traces}A,B for both materials respectively. Even though both light sources used for transmittance and reflectance spectra in transmission and epi-illumination mode should be the same halogen bulbs, the reflectance spectra suffered from greater noise in the $\leq \SI{1.6}{\electronvolt}$ range due to the lower intensity of light emitted by the epi-illumination bulb in this region. For this reason the reflectance spectra proved less suited to identifying flake thickness and are thus omitted from this work.
All spectra were collected using a $100\times$-magnification $0.55NA$ objective and \SI{150}{\micro\meter} thick core glass fibre leading to the spectrometer. The spectrometer collected and averaged 50 spectra that were integrated over \SI{500}{\milli\second}. To improve the rejection of stray light, the field aperture of the light was closed fully.
As can be seen in the transmittance spectra, the overall intensity difference is the best indicator for flake thickness beyond 2 layers whereas the location of the A-exciton is the key differentiator between 1 or 2 layers.
Using the spectroscope and the new microscope has greatly decreased the time required and greatly increased the ease of finding thin material and verifying its thickness.

\subsubsection{Assembling a heterostructure}
\begin{figure}[h]
	\centering
	\def\svgwidth{1\linewidth}
	\import{resources/Figures}{stamping.pdf_tex}
	\caption{Schematic outline of the stamping process. 1: Before starting, two PDMS stamps with monolayer material and a silicon substrate coated with PVA should be prepared. 2: The first flake is stamped, preferably as close to the centre of rotation of the substrate stage as possible as this makes aligning the next flake easier. 3: The micromanipulator with the PDMS stamp is slowly raised to peel the PDMS stamp off of the substrate, leaving behind the first monolayer flake. 4: The second stamp with monolayer material is aligned by rotating the substrate stage and moving the stamping micromanipulator. 5: The stamp is then carefully lowered and misalignment is corrected if necessary. 6: The second stamp is removed slowly, and both flakes are transferred to the substrate.}
	\label{fig:stamping_process}
\end{figure}

Once two suitably matching monolayer flakes have been located on different PDMS stamps they can be assembled to form a heterostructure.
The flakes are stamped onto a specially prepared substrate using a dedicated stamping set-up that allows for micrometer precise control over flake position and consists of two main parts: a substrate stage, and, a flake stage.
The substrate stage has three degrees of freedom: two micromanipulators control the $x$- and $y$-position of the substrate under the microscope and the stage itself is free to rotate to align the edges of the to be stamped ML flakes.
The stamping stage is capable of precisely moving a glass slide with a stamp along three axes.
The complete stamping set-up is pictured in Figure \ref{fig:stamping_set-up} and is a direct adaptation from previously published set-ups \cite{castellanos-gomezDeterministicTransferTwodimensional2014, castellanos-gomezDeterministicTransferTwodimensional2014a}.
The stamping set-up consists of: a reflective optical microscope, a rotating substrate stage attached to two micromanipulators, and, a stamping stage connected to three micromanipulators.
The ML material is stamped onto the substrate by first adhering the PDMS stamp onto a glass microscope slide and inserting this slide into a holding mechanism on the stamping stage, after which the stamp is carefully brought into contact with the substrate as illustrated in Figure \ref{fig:stamping_process}.
Since we wish for the flake to adhere to the substrate and not the stamp, we now slowly release the stamp from the substrate, the stamp peel-off speed is crucial at this step and should be as slow as possible without stopping the process.
During this step of the process it is also paramount to not introduce any vibration or other forces as these can easily tear the flake.
Applying the second stamp with ML material requires more care and effort as this stamp needs to be aligned with the first already stamped monolayer flake. The aligning step is performed by first locating the ML material on the substrate and then roughly aligning the micromanipulator with the second flake. Then experience has shown that the easiest way to proceed is by aligning the edges of both flakes to the desired angle as changing the angle of the flake on the substrate requires rotating the stage and most likely the stamped flake out of view. After alignment of the angles it is now possible to place the flakes above one-another using the micromanipulators. Slowly lowering the second stamp to the substrate will bring both flakes into view allowing for a more precise alignment, during further lowering of the flake its best to zoom out as to be able to see where the PDMS stamp first contacts the substrate. If this first point of contact is too far from the ML material, further lowering of the stamp can squeeze the PDMS and cause the flake to move in the direction of the contact-front propagation during further lowering; if this happens restarting the stamping procedure and compensating for the movement is possible, but it is advised to remove the glass slide with the flake completely and cut the PDMS in such a way that the first point of contact is closer to the ML material.
After stamping of the second flake is completed the PDMS needs to be slowly peeled of again in the same manner as before, after which the result will be a heterostructure on the prepared substrate, ready for transfer to a TEM grid.

\subsubsection{Transferring to a TEM grid}
\begin{figure}[h]
	\centering
	\def\svgwidth{1\linewidth}
	\import{resources/Figures}{wet_transfer.pdf_tex}
	\caption{Schematic illustration of the transfer process. 1: A previously prepared heterostructure is located on the substrate and suitable TEM grid is placed on the substrate over the heterostructures with tweezers. 2: The TEM grid is held in place, for the samples transferred for this work the TEM grid was held in place by pressing a microscope slide onto the rim of the TEM grid with the stamping micromanipulator taking care in not pressing too hard as this will push the TEM grid into the PVA sealing the inside in such a way that the IPA cannot reach the inside. 3: After wetting the grid with a drop of IPA the resulting surface tension will hold onto the carbon film as the drop evaporates pulling it onto the PVA coated substrate and heterostructure. 4: After all the IPA has evaporated and the glass slide is removed a single drop of distilled water is added to dissolve the PVA releasing the TEM grid and allowing it to float on the drop, ready to be picked up with tweezers.}
	\label{fig:wet_transfer}
\end{figure}

Transfer of the heterostructure to a TEM grid was performed using a previously devised polymer assisted method \cite{kosterPolymerassistedTEMSpecimen2021}, where a diced silicon wafer is coated using a polymer before stamping.
The flakes are then stamped directly on the polymer coating in the same manner as described in the previous section before being covered with a TEM grid, application of a single drop of either water or IPA  depending on the polymer used connects the flexible holey carbon film of the TEM grid with the polymer coating by means of surface tension as the drop evaporates. This "seals" the heterostructure in between the holey-carbon film and the polymer.
Adding another drop, this time the solvent for the polymer, allows the flake held by the TEM grid and the grid itself to separate from the silicon substrate; allowing it to be picked up using tweezers.
The process is illustrated in Figure \ref{fig:wet_transfer}.
The authors of the previously cited paper demonstrated that both the use of PMMA and PVA are possible but require different solvents, acetone and water respectively.
\marginnote{TODO: is this still accurate at the time of submission?}
The heterostructures transferred for this thesis were stamped on a diced silicon wafer coated with \(\sim \) \SI{100}{\nm} thick PVA and were then annealed at \SI{400}{\degreeCelsius} in vacuum as there was significant contamination present.


\newpage
\section{Advanced electron microscopy techniques to map moiré physics at the nanoscale; an electron microscope pixelated array detector.}

\subsection{Electron microscope pixel-array detector}
%In TEM operating mode a parallel electron beam is used to illuminate the sample and form an image on either the phosphorous screen or digital camera, this spreads the beam over a larger area such that the local electron dose is relatively uniform.

\begin{figure}[h]
	\centering
	\def\svgwidth{0.5\linewidth}
	\import{resources/Figures}{empad-haadf.pdf_tex}
	\caption{}
	\label{fig:empad_haadf_comparison}
\end{figure}
The electron microscope pixel-array detector, colloquially called the EMPAD for short, is as the name suggest a sensor for electron microscopes that consists of a grid of direct electron detecting pixels. Even though the EMPAD can be used in standard transmission electron imaging its strengths lie in scanning-TEM imaging due to the relatively little pixels but greater dynamic range of said pixels.
In a scanning-TEM (STEM) mode the beam is focused to a small point that scans over the specimen. Elastic scattering then deflects the electrons from the optical axis onto one of the ring shaped detectors encircling the optical axis, these detectors are called the dark field detectors. These singular annular detectors bunch the elastically scattered electrons together, losing valuable information on the exact scattering angle. The electron microscope pixel-array detector (EMPAD) solves this problem by replacing the annular dark field and singular bright field detectors with a single fast-readout, high dynamic range pixel grid on which every pixels' electron dose is stored separately such that after acquisition of a complete STEM scan the bright- and dark-field detectors can be virtually recreated by integrating the electron dose using annular or circular masks on the data.
The pixelated nature of the detector enables the precise computation of the intensity distribution of not only the whole CBED patten but also within each diffraction disk within the CBED pattern, greatly improving the potential resolution of ptychography methods \cite{pennycookEfficientPhaseContrast2015, yangEfficientPhaseContrast2015a} as well as enabling charge density analysis \cite{hachtelSubAngstromElectricField2018,wenMapping1DConfined2022,fangAtomicElectrostaticMaps2019}.
A greater accuracy in scattering angle is also available since every pixel in the array has its own smaller range of scattering angles whose electrons the pixel collects.


\begin{figure}[h]
	\centering
	\def\svgwidth{1\linewidth}
	\import{resources/Figures}{4d-dataset.pdf_tex}
	\caption{}
	\label{fig:4d_dataset}
\end{figure}

\subsection{Strain analysis}
\subsubsection{Micrometer field-of-view strain analysis}
\subsubsection{Nanometre field-of-view strain analysis}

\subsection{Ptychography}

\subsection{Charge density analysis}




