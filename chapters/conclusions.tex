\section{Conclusions and Outlook}
\label{sec:outlook}
The stamping and mechanical transfer steps outlined in this report have been demonstrated to have produced at least a single large scale \SI{1}{\micro\meter} heterobilayer structure whose moiré angle was not yet controlled. This has shown that the method, although tedious and once improved upon, has the potential for rapid prototyping of heterostructures with varying material makeup, material thicknesses, moiré twist angle, and possibly in the future even strain. The feasibility of the spectroscopy set-up to identify material and its thickness has been proven to be sufficient for the characterisation and selection of TMD flakes. The wet transfer steps first attempted in this group for the transfer of heterostructures from silicon to holey carbon grids have been improved with time and experience, so that going forward a higher success rate can be achieved.\\
The EMPAD has been successfully employed to characterise the heterostrain in both stacked layers of a heterostructure with a moiré angle of \SI{12.2}{\degree} for a large field of view. The strain analysis technique utilising the exit-wave power cepstrum transform has proven powerful in dealing with unfavourable features commonly found in stamped and stacked heterostructures such as substrates and sample tilt. The centre-of-mass analysis on the EMPAD data has opened up a new possibility for characterising our stamped heterostructures' electric and magnetic potentials at the moiré length scales and was able to disentangle the signal coming from differently sized features. Further thought and experiments will be needed to gain the necessary expertise and intuition required to effectively select the feature size whose data is captured at the time of acquisition.

Overall, the processes implemented in this report lay the groundwork for the prototyping of new heterostructures and find the link between physical characteristics, like strain and twist, and the electronic and magnetic properties of these materials. With the future help of machine learning to better analyse the vast quantities of data gathered by the EMPAD, it will be possible to design and manufacture more sophisticated quantum systems.
