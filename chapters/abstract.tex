\chapter*{Abstract}
%
In the realm of material science, the study of moiré heterostructures, particularly those formed from van der Waals materials such as MoSe$_2$ and WSe$_2$, has gained significant attention due to their unique properties and potential applications. 
%
This research focuses on the application of scanning transmission electron microscopy (STEM) enhanced by a state-of-the-art direct electron detector with a high dynamic range. 
%
Our aim is to acquire precise, localised information about the strain and potential fields within these moiré structures with unprecedented accuracy. 
%
Employing an innovative electron microscope pixelated array detector, we captured diffraction data from small areas of heterostructure lattices, generating comprehensive four-dimensional datasets. 
%
These datasets enable us to map strain and electrical properties at the micrometre and nanometre scale with remarkable precision. 
%
A novel aspect of our methodology is the development and integration of a centre-of-mass analysis approach into an existing dataset exploration framework, facilitating rapid experimentation and data analysis during electron microscope use. 
%
The findings of this study not only contribute to the advancement of moiré heterostructures analysis but also pave the way for future innovations in microscopy techniques and material science. 
%
Our research highlights the potential of cutting-edge microscopy in unraveling the complexities of advanced materials, offering insights that could lead to breakthroughs in nanotechnology and electronics.